\subsection{Test preparation}
\label{performance_evaluation:testpreparation}
%
The target of such tests was to verify \Item{i} if, using 
an exocentric vision system provided with such an algorithm, 
users are able to perceive the trajectory actually 
performed by the robot, \Item{ii} how much disturbing are 
sudden point-of-view changes for users, \Item{iii} how 
comfortable is for them to view a robot from a 
virtual exocentric point-of-view.
\\
People involved in the tests have had no experience in 
robotics, neither they knew what a virtual exocentric 
vision system is.
\\
The testing session schedule was the following:
\begin{enumerate}
  \item testers were given a brief introduction to exocentric vision systems
  \item each tester is given three different scenarios and has to 
    complete them, unassisted
  \item for each scenario, testers have to answer the questions 
    of a questionnaire\footnote{a copy of the questionnaire is 
      reported in this document, in section \ref{sec:appendix_a}}
\end{enumerate}

Three different recorded logs have been used, each of which featuring 
a different trajectory. Each of such session has been tested
using three different \texttt{mu\_distance} values: 5, 15 and 25.
Results are shown in next section,
\ref{performance_evaluation:tests_result}.
