\setcounter{figure}{0}
\setcounter{table}{0}
\setcounter{lstlisting}{0}

\chapter{Future works}
\label{future_works}
\minitoc

This section contains some tips for the future development of 
\framework{} and the \textit{sweep metric algorithm}.
\\
It would be good to make a comparison between the \textit{sweep angle
algorithm} and selection method no. 3 described in \cite{sugimoto}.
\\
The formula used in \cite{sugimoto} to compare the view-point of an image with
the robot's current position and direction is not clear at all. Some parameters
are ambiguous and there is not an exhaustive explanation about the formula that
ties them together.
\\
After resolving and implementing the formula, it could be interesting to evaluate
the same case tests with the two approaches, in order to underline the advantages
and the disadvantages shown by each method and compare them.
\\
New implementations of the \texttt{IDataLogic} interface could be 
developed. One could, for instance, interact by a socket or another 
data stream) with the simulator making the whole system working \textit{online}, 
without the need for saved log files and images. Another one could exploit the 
same approach to connect to the actual \morduc{}. It would also make possible 
to actually operate the robot remotely.
\\
In both cases, the class implementing \texttt{IDataLogic} must open a communication
channel, in order to send commands to the robot and retrieve its data.
\\
If the server contacted is the \morduc{} itself, the concrete \texttt{IDataLogic} 
implementation must know the \morduc{} communication protocol (see section 2.4 of 
\cite{morduc:dasero}). Otherwise, if the client communicates with a custom 
simulator, the communication protocol could be chosen by the developer.
\\
In document \cite{morduc:neri}, Neri and others prove (by several tests) that \textit{3D
vision guarantees a major precision in the teleguide and good performances on the obstacles
avoidance}. An interesting future development could add the stereoscopic vision to a concrete
\framework{} application, in order to merge the advantages brought by the two different
approaches in robot teleguiding.
\\
To implement 3D vision, the robot server must provide both the right and left camera images,
whereas the OpenGL functions must draw robot in the proper way on the images retrieved, to render
the 3D effect. More details depends on the 3D technologies chosen by the developer: shutter-glasses,
anaglyph, polarized, or other.
\\
The work presented in this document dis not take into account 
collisions. If the environment the robot moves in presents walls 
or obstacles to avoid, it could be useful to advise user in case 
of an imminent or already happened collision.
\\
A signalling system could be implemented once again with OpenGL, therefore 
with augmented reality.The laser value, provided by the \morduc{},
could help to understand when and where draw warnings, as shown in 
\cite{morduc:macalusodetommaso}.
\\
A more simple, but doubtless useful, future upgrade would consists in 
creating a graphical interface, which allows user to define the
\textit{sweep metric algorithm} initial parameters in a more friendly way. 
Currently, all these parameters (e.g. the number of log session or the 
optimal distance) are given by command line.
\\
Last, but surely not least, two suggestions for the 
\textit{sweep metric algorithm} reported in section 
\ref{performance_evaluation:finalconsiderations}.
\\
An upgraded version of the algorithm could feature a 
\textit{variable} optimal distance value. That value 
could, for instance, be increased when the robot is moving 
along straight lines and could be decreased when it starts 
turning.
\\
To avoid sudden point-of-view changings during hard turnings, 
a possible solution would be to modify the image selection algorithm 
to have it behaving as follows: when it detects hard turnings, the 
image to be set as background should provide a greater field-of-view. 
In other words, instead of selecting the image provided by the 
egocentric camera, the algorithm should select an image taken 
in a position more distant than the optimal distance.
