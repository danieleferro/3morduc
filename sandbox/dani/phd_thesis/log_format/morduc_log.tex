\section{Morduc Log}
\label{log:morduc}

Different working groups have used \morduc{} to implement their
specific client. A large amount of log data
information has so been collected from real sessions, in
order to replicate offline all the state hold by the robot
during previous teleguiding runs.
\\
Because \morduc{} server does not implement any functionality
to save log data on its disk, clients which wished to store
images and data collected during online runs have developed their
proper piece of code to create log files locally.
Even though every client can create log file with a proper and
specific format, every one followed a specific but not official
standard, described in this section.
\\
Using a common way to store data allows older and future client to
test their functionality on a wide set of data log, without
worrying about which client collected the information:
changing the client does not require to change data log,
thanks to the standard format.
\\
Another advantages of this approach 
its that the server is not overloaded
with log data, because they are stored locally on client.
\\
For our tests we can rely on two text files and several images.
Usually client add new data to the log file set 
when robot changes its status,
i.e. its position, because a teleguide command has
been received.
\\
Cameras' image are saved in 
a single 1280x480 pixel JPEG image, composed by two
640x480 pixel images, one from the right and one from the left camera.
A counter variable,
starting from number one and incremented every time a new image
has to be written, is used in order to assign each one a different
name according to the following notation:

\begin{center}
  \texttt{img\_$\langle$counter\_value$\rangle$.txt}
\end{center}

Every created image is coupled with two different type of
information, collected in the shooting instant and providing,
respectively, robot's odometric and scanner laser data.
\\
The former is saved the text file named

\begin{center}
  \texttt{odometric.txt}
\end{center}

where the line i-th contains the odometric data tied to the i-th
images. Every line is copied from the HTTP response header sent
by the \morduc{} server (see section \ref{intro:3morduc:communication}),
so the format will be

\begin{center}
  \texttt{Morduc/$\langle$time$\rangle$$\backslash$$\langle$x$\rangle$$\backslash$$\langle$y$\rangle$$\backslash$$\langle$theta$\rangle$$\backslash$$\langle$collision\_number$\rangle$$\backslash$$\langle$min\_distance$\rangle$}
\end{center}

obviously, the text file will contain as many lines as many
saved images.
\\
Laser data are saved, instead, in file named

\begin{center}
  \texttt{laser.txt}
\end{center}

also in this case line i-th is coupled with the i-th image, and
every line is rewritten from text data read in HTTP header
response:

\begin{center}
  \texttt{Laser/$\langle$value\_1$\rangle$/$\langle$value\_2$\rangle$/../$\langle$value\_181$\rangle$}
\end{center}

Since laser scan 180$\textdegree$ with 1$\textdegree$ step,
181 values are saved.
\\
Even though \framework{} do not takes
advantages of laser information, this could be used by future
version in order to improve operator skill in guiding the robot
(e.g. by indicating distance from nearest object with augmented
reality, as implemented in \cite{morduc:macalusodetommaso}).
