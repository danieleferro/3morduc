\section{Download, compile and run \framework{}}
\label{sourcecode:downloadrun}

\subsection{Download code}
\label{sourcecode:downloadrun:download}

The \framework{} project is hosted on \textit{Google Code}:
\\
\\
\url{http://code.google.com/p/3morduc/}
\\
\\
New \framework{} version, covered by this document,
can be downloaded at the following URL, together with
log files created with \morduc{}'s simulator and from
real \morduc{} online session:
\\
\\
\url{http://3morduc.googlecode.com/files/rear.thesis.tar.gz}
\\
\\
This report itself can be found at:
\\
\\
\url{http://3morduc.googlecode.com/files/thesis_daniele_ferro.pdf}
\\
\\
The first release of the source code, together with
some relative log files, can be obtained by downloading
the tar archive named \textit{rear.tar.gz}, from the
following URL:
\\
\\
\url{http://3morduc.googlecode.com/files/rear.tar.gz}
\\
\\
while the related document can be found at:
\\
\\
\url{http://3morduc.googlecode.com/files/report.14.07.pdf}

\subsection{Compile code}
\label{sourcecode:downloadrun:compile}

The instructions exposed in this section are also valid for the
first version of \framework{}, since it utilizes a restricted
set of libraries.
\\
\framework{} is written in C++ and makes use of OpenGL commands.
Hence, in order to use it, you need to install a C++ compiler and the
C++ standard library, together with an implementation of the OpenGL
API and some free libraries.
\\
In particular, \framework{} has been developed using different versions
of GNU \textit{gcc} as
C++ compiler (4.3.2, 4.4.1 and 4.4.5) and different versions of 
\textit{Mesa} as the
implementation
of OpenGL (7.0.3, 7.6.0, and 7.7.1).
\\
A list of other required libraries follows:

\begin{itemize}

  \item \texttt{freeGLUT} \\
    \textit{Description:} \small {GLUT allows the user to create and manage
      windows containing OpenGL contexts and also read the mouse, keyboard
      and joystick inputs.} \\
    \textit{Site:} \url{http://freeglut.sourceforge.net/} \\
    \textit{Tested with version:} 2.4.0, 2.6.0

  \item \texttt{libPNG} \\
    \textit{Description:} \small {Official PNG reference library, supporting
      almost all PNG features.} \\
    \textit{Site:} \url{http://www.libpng.org/pub/png/libpng.html} \\
    \textit{Tested with version:} 1.2.27, 1.2.44

  \item \texttt{libJPEG} \\
    \textit{Description:} \small {Widely used free library for JPEG image
      compression.} \\
    \textit{Site:} \url{http://www.ijg.org/} \\
    \textit{Tested with version:} 62 \\
    \textit{Note}: \small{Since version 62 does not contain \texttt{jinclude.h},
      which defines few but essential macros, the latter has been included in
      \framework{} source code. This assures compatibility with the stable
      libJPEG version (62) as well as the most recent ones.}

  \item \texttt{cURLpp} \\
    \textit{Description:} \small {A C++ wrapper for \texttt{libcURL}, a free and
      easy-to-use client-side URL transfer library, supporting FTP, FTPS, HTTP, HTTPS
      and other protocols.} \\
    \textit{Site:} \url{http://curlpp.org/} \\
    \textit{Tested with version:} 0.7.3 \\
    \textit{Note}: \small{After compiling and installing the library, still not present
      in Debian repository, you need to add \texttt{/usr/local/lib} directory
      to your library path. For this purpose you can add the directory in
      \texttt{/etc/ld.so.conf}, and the run
      \texttt{ldconfing} as superuser to active new path.}

\end{itemize}

After installing all the previous libraries, \framework{}
source code can be compiled. In order to make this
task easier, a \texttt{Makefile} for Unix system
is available, assuming source code is stored in
\texttt{source} folder, whereas object files in - 
temporanely - \texttt{obj} folder.
\\
Accepted \texttt{make} commands are: 

\begin{itemize}

  \item \texttt{\$ make} \\
    Compile source code and produce \texttt{REAR} executable
    file.

  \item \texttt{\$ make clean} \\
    Delete every object files and the executable one.

  \item \texttt{\$ make online}, \texttt{\$ make log}, \texttt{\$ make logsim} \\
    Run \framework{} with common parameters, to execute, respectively,
    an online run with \morduc{}, a run based on previous \morduc{}'s log files,
    a run based on log created with \morduc{}'s simulator.

\end{itemize}

\texttt{Makefile} can be easily edited to obtain a corresponding
Windows based version.


\subsection{Run \framework{}}
\label{sourcecode:downloadrun:run}

\framework{} takes several parameters in input, to let
user specify with which features the main program must be called. 
\\
A complete description of possible options to state is
shown with command:
\\
\\
\texttt{./REAR -h}
\\
\\
while the following general usage is displayed when a wrong options sequence
or incorrect values are submitted:
\\
\\
\texttt{Usage: REAR -dl data\_logic dl\_opt1 [dl\_opt2]
  [-is image\_select is\_dist] [-r radius]}
\\
\\
Let's examine what value each option can assume.
\\
The unique mandatory option is the one defining which type
of \textit{DataLogic} instance has to be instantiated. The
\texttt{-dl} token must always be expressed after the
executable file name, followed by \textit{DataLogic} identifier
and at least one option.
\\
Based on the value assumed by \texttt{data\_logic}, one or
two options can be specified. All possible combinations
are showed in table \ref{table:dloptions}.

\begin{table}[!h]
  \centering  
  \begin{tabular}{| c | c | c | c |}

    \hline
    \texttt{data\_logic} &
    \texttt{dl\_opt1} &
    \texttt{dl\_opt2} &
    \texttt{example} \\

    \hline
    \small {morduc} &
    \small {$<$IP\_or\_URL$>$} &
    \small {$<$data\_path$>$} &
    \small{-dl morduc 192.168.1.2 ../myfolder} \\

    \hline
    \small {logmorduc} &
    \small {$<$log\_number$>$} &
    \small {-} &
    \small{-dl logmorduc 5} \\


    \hline
    \small {logsimul} &
    \small {$<$log\_number$>$} &
    \small {-} &
    \small {-dl logsimul 2} \\

    \hline

  \end{tabular}
  \caption{Possible values for \texttt{-dl} options.}
  \label{table:dloptions}
\end{table}

By indicating \texttt{morduc} as \texttt{data\_logic}, a instance
of \texttt{DataLogicMorduc} is created (section
\ref{concr:idatalogic:datalogicmorduc}). Needed information are
the server's IP or URL to connect to and directory path where images
and odometric data will be saved as they are received from \morduc{}.
If the latter is not specified they are saved in 
\\
\\
\texttt{../log\_morduc/log\_online/}
\\
\\
starting from the executable file path.
\\
Instead, if \texttt{logmorduc} or \texttt{logsimul} options are
specified, data are retrieved from previously stored image and text
files  (further details can be found in chapter \ref{log}).
In first case logs were created from previous real \morduc{}'s
teleguiding session and an instance of \texttt{DataLogicLogMorduc}
is used (section \ref{concr:idatalogic:datalogiclogmorduc});
in second one logs were created with \morduc{}'s simulator and
an object of type \texttt{DataLogicLogSimulator} is instantiated
(section \ref{concr:idatalogic:datalogiclogsimulator}).
\\
Since both type of log files must be saved in a specific directories and
are identified by an integer number, their unique option allows to indicate
which set of log files must be used.
\\
After the \texttt{-dl} option, user can state other two optional
aspects, in any order. One regards what kind of \textit{Image Selector
Algorithm} must be used, along with one single options which defines
its \textit{optimal distance}, i.e. the distance between robot and
the exocentric point of view that ensures the better performance in
teleguiding the robot.
\\
As covered in previous chapters, three algorithms have been developed,
so the total number of possible values for the \texttt{-is} option
is three, as table \ref{table:isoptions} summarizes.

\begin{table}[!h]
  \centering  
  \begin{tabular}{| c | c | c | c |}

    \hline
    \texttt{image\_select} &
    \texttt{is\_dist} &
    \texttt{example} \\

    \hline
    \small {spacial} &
    \small {$<$opt\_distance$>$} &
    \small{-is spacial 15} \\

    \hline
    \small {sweep} &
    \small {$<$opt\_distance$>$} &
    \small{-is sweep 20} \\

    \hline
    \small {asweep} &
    \small {$<$opt\_distance$>$} &
    \small{-is asweep 10} \\

    \hline

  \end{tabular}
  \caption{Possible values for \texttt{-is} options.}
  \label{table:isoptions}
\end{table}

In order, a \texttt{SpacialMetricCalc} instance
(section \ref{concr:iimageselector:spacial_metric_class}),
a \texttt{SweepMetricCalc} instance
(section \ref{concr:iimageselector:sweep_metric_class}) or
a \texttt{AnotherSweepMetricCalc} instance
(section \ref{concr:iimageselector:another_sweep_metric_class})
is created to exploit the image selection algorithm encapsulated
in each one. If \texttt{-is} is not specified, the latter class
is used as default, with optimal distance value set to twentyfive.
\\
Last possible customization can be obtained by stating the
\morduc{}'s radius value, used to draw in OpenGL space the three
vertical disks composing the robot. After several tests, the
most performance value when using data from real \morduc{} robot
(retrieved online or from log files) has been proved to be five; 
instead, if images and data come from simulator, a better
values for \morduc{} radius is four.
\\
If not specified with the \texttt{-r $<$radius$>$} option, robot
will be drawn with a radius equal to five.
\\
At last, table \ref{table:validopts} shows some complete example.

\begin{table}[!h]
  \centering  
  \begin{tabular}{| l | l |}

    \hline
    \texttt{command} &
    \texttt{description} \\

    \hline
    \small {REAR -dl morduc 192.168.1.2 -is asweep 25} &
    \parbox[t]{6cm}{\raggedright \small
      Connect to real \morduc{} with address 192.168.1.2, draw robot
      with radius equal to 5 (default) and use \textit{Another Sweep
      Metric Algorithm} with optimal distance set to 25.} \\  [1ex]

    \hline
    \small {REAR -dl logmorduc 4 -r 4 -is sweep 10} &
    \parbox[t]{6cm}{\raggedright \small
      Use \morduc{}'s log number four, draw robot
      with radius equal to four and use \textit{Sweep
      Metric Algorithm} with optimal distance set to 10.} \\  [1ex]

    \hline
    \small {REAR -dl logsimul 5} &
    \parbox[t]{6cm}{\raggedright \small
      Use simulator's log number five, draw robot
      with radius equal to five (default) and use \textit{Another Sweep
      Metric Algorithm} with optimal distance set to 25 (default values).} \\  [1ex]


    \hline

  \end{tabular}
  \caption{Some valid command inputs.}
  \label{table:validopts}
\end{table}
