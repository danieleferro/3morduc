\section{Morduc Log}
\label{log:morduc}

Different working groups have used \morduc{} to implement its
specific client. A large amount of log data
information has so been collected from real sessions, in
order to replicate offline all the state hold by the robot
during previous teleguiding sessions.
\\
We can rely on two text files and several images. Every time
robot changes its position, because a teleguide command has
been received, new data are added to the log file set.
\\
The camera images, from right and left camera, are saved in 
a 1280x480 pixels JPEG image. A counter variable,
starting from number one and incremented every time a new image
has to be written, is used in order to assign each one a different
name according to the following notation:

\begin{center}
  \textit{img\_$\langle$counter\_value$\rangle$.txt}
\end{center}

Every created image is coupled with two different type of
information, collected in the shooting instant and provided,
respectively, by robot odometric data and scanner laser.
\\
The former is saved the text file named

\begin{center}
  \textit{odometric.txt}
\end{center}

where the line i-th contains the odometric data tied to the i-th
images. New text is copied from the HTTP response header sent
by the \morduc{} server (see section \ref{intro:3morduc:communication}),
so the format will be

\begin{center}
  \textit{Morduc/$\langle$time$\rangle$$\backslash$$\langle$x$\rangle$$\backslash$$\langle$y$\rangle$$\backslash$$\langle$theta$\rangle$$\backslash$$\langle$collision\_number$\rangle$$\backslash$$\langle$min\_distance$\rangle$}
\end{center}

obviously, the text file will contain as many lines as many
saved images.
Laser data are saved, instead, in file named

\begin{center}
  \textit{laser.txt}
\end{center}

also in this case line i-th is coupled with the i-th image, and
every line is rewritten from text data read in HTTP header
response:

\begin{center}
  \textit{Laser/$\langle$value\_1$\rangle$/$\langle$value\_2$\rangle$/../$\langle$value\_181$\rangle$}
\end{center}

Inserted data are 181 values, since laser scan 180 \textdegree
with 1 \textdegree step. Even though \framework{} do not takes
advantages of laser information, this could be used by future
version in order to improve operator skill in guiding the robot
(e.g. by indicating distance from nearest object with augmented
reality, as proposed in \cite{morduc:macalusodetommaso}).
