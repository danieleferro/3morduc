\setcounter{figure}{0}
\setcounter{table}{0}
\setcounter{lstlisting}{0}

\chapter{Source code}
\label{sourcecode}
\minitoc

\framework{} was developed and released
by \textbf{Daniele Ferro}\footnote{mail: daniele.ferro86
at gmail dot com}
and
\textbf{Loris Fichera}\footnote{mail: loris.fichera
at gmail dot com}, after
a foreign study
period spent at the University of Hertfordshire (United Kingdom).
\\
Back in our country, an improvement version of \framework{} has
been written.
New code, and this document itself, are based on the primary
work done in U.K., presenting the following features in
addition:

\begin{itemize}

\item working with log data saved from previous
      \morduc{} session. \\
      See section \textit{DataLogicLogMorduc},
      \ref{concr:idatalogic:datalogiclogmorduc}.
      

\item connecting to real \morduc{} robot to implement
      an online teleguiding session. \\
      See section \textit{DataLogicMorduc},
      \ref{concr:idatalogic:datalogicmorduc}.

\item implementing a new version of \textit{Sweep Metric
      Algorithm}, to overcame deficiencies detected in
      performed tests. \\
      See sections \textit{Another sweep metric algorithm}
      an \textit{AnotherSweepMetricCalc class},
      \ref{concr:iimageselector:another_sweep_metric_algorithm},
      and
      \ref{concr:iimageselector:another_sweep_metric_class}.

\end{itemize}

Other minor changes to the original C++ code have occurred, such
as source reorganization in new modules; some classes have
been renamed to better identify their role.
\\
Project has been developed and tested with \textit{Debian Lenny (5.0)}
and \textit{Debian Squeeze (6.0)}. In the following sections major
details about packages and free libraries utilized are given.
\\
\framework{} is distributed under \textit{GNU GPL v3} license, 
documented in \cite{license:gplv3}. 
Briefly, General Public Licenses are designed to give these rights: to run the
program, for any desired purpose; to study how the program works, and
modify it; to redistribute copies and to improve the program, and
release the improvements to the public.
\\
Next section will explain how to download, compile and
run \framework{}. In second section instead a brief user guide
is provided.

\clearpage
\section{Download, compile and run \framework{}}
\label{sourcecode:downloadrun}

This section will explain how to download, compile and
run \framework{}. At last, a user guide to use the
framework is provided.

\subsection{Download the code}
\label{sourcecode:downloadrun:download}

The \framework{} project is hosted on \textit{Google Code}:

\begin{center}
  \url{http://code.google.com/p/3morduc/}
\end{center}

The first release of the source code, together with
some saved log files, can be obtained by downloading
the tar archive named \textit{rear.tar.gz}, from the
following URL:

\begin{center}
\url{http://3morduc.googlecode.com/files/rear.tar.gz}
\end{center}

while the related document can be found at:

\begin{center}
\url{http://3morduc.googlecode.com/files/report.14.07.pdf}
\end{center}

New \framework{} version, this document concerns about,
can be downloaded from the following URL, together with
log files from \morduc{} simulator and from different \morduc{}
clients:

\begin{center}
\url{http://3morduc.googlecode.com/files/rear.thesis.tar.gz}
\end{center}

At last, this report its self can be found at:

\begin{center}
\url{http://3morduc.googlecode.com/files/thesis_daniele_ferro.pdf}
\end{center}

\subsection{Compile the code}
\label{sourcecode:downloadrun:compile}

The instruction exposed in this section are also valid for the
first version of \framework{}, since it utilize a restricted
set of libraries.


\clearpage
\section{User guide}
\label{sourcecode:userguide}

Table \ref{table:usercommands} shows user's commands and their descriptions.

\begin{table}[!h]
  \centering  
  \begin{tabular}{| l | l |}

    \hline
    \texttt{command key} &
    \texttt{description} \\

    \hline
    \small {ESC} &
    \parbox[t]{9cm}{\raggedright \small
      Close window and exit program.} \\  [1ex]

    \hline
    \small {A} &
    \parbox[t]{9cm}{\raggedright \small
      Send command to go forward. If data log are used, specific command is ignored
      and next robot's status is retrieved.} \\  [1ex]

    \hline
    \small {S} &
    \parbox[t]{9cm}{\raggedright \small
      Send command to go backward. If data log are used, specific command is ignored
      and next robot's status is retrieved.} \\  [1ex]

    \hline
    \small {Q} &
    \parbox[t]{9cm}{\raggedright \small
      Send command to turn left. If data log are used, specific command is ignored
      and next robot's status is retrieved.} \\  [1ex]

    \hline
    \small {W} &
    \parbox[t]{9cm}{\raggedright \small
      Send command to turn right. If data log are used, specific command is ignored
      and next robot's status is retrieved.} \\  [1ex]

    \hline

  \end{tabular}
  \caption{User keyboard commands.}
  \label{table:usercommands}
\end{table}

Other commands are active for debugging purpose, listed
in table \ref{table:debugcommands}. Calling one or more of
these commands will not affect the system response to next
user's command.

\begin{table}[!h]
  \centering  
  \begin{tabular}{| l | l |}

    \hline
    \texttt{command key} &
    \texttt{description} \\

    \hline
    \small {O} &
    \parbox[t]{9cm}{\raggedright \small
      Move camera object along the X axis (positive sense).} \\  [1ex]

    \hline
    \small {P} &
    \parbox[t]{9cm}{\raggedright \small
      Move camera object along the X axis (negative sense).} \\  [1ex]

    \hline
    \small {K} &
    \parbox[t]{9cm}{\raggedright \small
      Rotate camera along Z axis (clockwise).} \\  [1ex]

    \hline
    \small {L} &
    \parbox[t]{9cm}{\raggedright \small
      Rotate camera along Z axis (counterclockwise).} \\  [1ex]

    \hline
    \small {$\uparrow$ (up arrow)} &
    \parbox[t]{9cm}{\raggedright \small
      Move robot along Y axis (positive sense).} \\  [1ex]

    \hline
    \small {$\downarrow$ (down arrow)} &
    \parbox[t]{9cm}{\raggedright \small
      Move robot along Y axis (negative sense).} \\  [1ex]

    \hline
    \small {$\rightarrow$ (right arrow)} &
    \parbox[t]{9cm}{\raggedright \small
      Move robot along X axis (positive sense).} \\  [1ex]

    \hline
    \small {$\leftarrow$ (left arrow)} &
    \parbox[t]{9cm}{\raggedright \small
      Move robot along X axis (negative sense).} \\  [1ex]

    \hline

  \end{tabular}
  \caption{Debug keyboard commands.}
  \label{table:debugcommands}
\end{table}

