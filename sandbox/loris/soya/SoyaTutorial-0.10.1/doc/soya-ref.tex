\section{Module: \module{soya}}
\declaremodule{standard}{soya}
\subsection{Functions}
\begin{funcdesc}{clear_events}{...}
go through all the events in the key and remove them.
mask currently on supports ALL\_EVENTS
\end{funcdesc}

\begin{funcdesc}{coalesce_motion_event}{events}
coalesce\_motion\_event(events) -> sequence

Prunes from EVENTS all mouse motion events except the last one.
This is usefull since only the last one is usually releavant (Though
be carrefull if you use the relative mouse coordinate !).

EVENTS should be a list of events, as returned by soya.process\_event().
The returned list has the same structure.
\end{funcdesc}

\begin{funcdesc}{cursor_set_visible}{...}
\end{funcdesc}

\begin{funcdesc}{do_cmd}{cmd}
\end{funcdesc}

\begin{funcdesc}{get_grab_input}{...}
queries mouse grabbing
returns 1 or 0
\end{funcdesc}

\begin{funcdesc}{get_max_texture_size}{...}
\end{funcdesc}

\begin{funcdesc}{get_mod}{...}
\end{funcdesc}

\begin{funcdesc}{get_mouse_rel_pos}{...}
return the relative mouse position since the last call to this function
\end{funcdesc}

\begin{funcdesc}{get_screen_height}{...}
\end{funcdesc}

\begin{funcdesc}{get_screen_width}{...}
\end{funcdesc}

\begin{funcdesc}{image_from_pil}{pil_image}
Creates and returns a Soya image from PIL image PIL\_IMAGE.
\end{funcdesc}

\begin{funcdesc}{init}{title = "Soya 3D", width = 640, height = 480, fullscreen = 0, resizeable = 1, create_surface = 1}
Inits Soya 3D and display the 3D view.

TITLE is the title of the window.
WIDTH and HEIGHT the dimensions of the 3D view.
FULLSCREEN is true for fullscreen and false for windowed mode.
RESIZEABLE is true for a resizeable window.
\end{funcdesc}

\begin{funcdesc}{load_raw_image}{...}
Loads a ".raw" image file, which are used by Cal3D example (see cal3d\_data).
Returns a Soya image object, suitable for model.material.image.
\end{funcdesc}

\begin{funcdesc}{open_image}{filename}
Creates and returns a Soya image from file FILENAME.
The image is loaded with PIL; the supported image formats are the same than PIL.
\end{funcdesc}

\begin{funcdesc}{parse_cal3d_cfg_file}{...}
Reads a the Cal3D .cfg file, and creates and returns a Cal3D shape from it.
\end{funcdesc}

\begin{funcdesc}{process_event}{...}
\end{funcdesc}

\begin{funcdesc}{quit}{...}
\end{funcdesc}

\begin{funcdesc}{render}{}
Renders the 3D scene. Use set\_root\_widget() to choose which camera is used.
\end{funcdesc}

\begin{funcdesc}{screenshot}{filename = None, x = 0, y = 0, width = soya.get_screen_width(}
Take a screenshot of the soya rendering screen, and return it as a PIL image object.
If given, the image is saved under FILENAME.
You must have rendered something before (e.g. by calling soya.render, or by running
an Idler).
\end{funcdesc}

\begin{funcdesc}{set_gamma}{...}
Defines gamma correction.
Usage : soya.set\_gamma(red\_gamma,green\_gamma,blue\_gamma)
if XXX\_gamma=1.0 : no change
>1.0 : dark
<1.0 : bright
\end{funcdesc}

\begin{funcdesc}{set_grab_input}{...}
grabs the mouse and keyboard input to our window
set\_grab\_state(0|1) to set state
\end{funcdesc}

\begin{funcdesc}{set_mouse_pos}{...}
move the mouse cursor to x,y
\end{funcdesc}

\begin{funcdesc}{set_quality}{...}
\end{funcdesc}

\begin{funcdesc}{set_root_widget}{widget}
set\_root\_widget(WIDGET)

Sets the root widget to WIDGET. The root widget is the widget that is rendered first.
It defaults to the first Camera you create.
\end{funcdesc}

\begin{funcdesc}{set_use_unicode}{...}
when set process\_event will return a 4 part tuple for a keydown event.
the fourth part contains the unicode symbol for the key.
usefull for getting CAPITALS and !�\$\% etc in text boxes for example
\end{funcdesc}

\begin{funcdesc}{set_video}{...}
\end{funcdesc}

\begin{funcdesc}{toggle_wireframe}{...}
\end{funcdesc}

\subsection{Classes}
\begin{classdesc*}{Atmosphere}
Inherits:
\class{_Atmosphere}
\class{_CObj}

Atmosphere

An Atmosphere is an object that defines all the atmospheric attributes of a World, such
as fog, background or ambient lighting.

To apply an Atmosphere to a World, as well as everything inside the World, do :

    world.atmosphere = my\_atmosphere

You can safely put several Worlds one inside the other, with different Atmospheres.

Attributes are :

 - fog: true to activate fog, false to disable fog (default value).

 - fog\_color: the fog color (an (R, G, B, A) tuple of four floats). Defaults to black.

 - fog\_type: the type of fog. fog\_type can take 3 different values :
   - 0, linear fog: the fog range from fog\_start to fog\_end (default value).
   - 1, exponentiel fog: the fog the fog increase exponentially to fog\_density and the distance.
   - 2, exponentiel squared fog: the fog the fog increase exponentially to the square of fog\_density and the distance.

 - fog\_start: the distance at which the fog begins, if fog\_type == 0. Defaults to 10.0.

 - fog\_end: the distance at which the fog ends, if fog\_type == 0. Defaults to 100.0.

 - fog\_density: the fog density, if fog\_type > 0. Defaults to 1.0.

 - ambient: the ambient lighting color (an (R, G, B, A) tuple of four floats). Defaults to (0.5, 0.5, 0.5, 1.0).

 - bg\_color: the background color of the scene (an (R, G, B, A) tuple of four floats). Defaults to black.

\end{classdesc*}

\begin{classdesc*}{Bonus}
Inherits:
\class{_Bonus}
\class{CoordSyst}
\class{Position}
\class{_CObj}

No documentation for this class.

\end{classdesc*}

\begin{classdesc*}{Cal3dShape}
Inherits:
\class{Shape}
\class{SavedInAPath}
\class{_Shape}
\class{_CObj}
\class{_Cal3dShape}
\class{_Shape}
\class{_CObj}

No documentation for this class.

\begin{datadesc}{DIRNAME}
\end{datadesc}
\end{classdesc*}

\begin{classdesc*}{Cal3dVolume}
Inherits:
\class{_Cal3dVolume}
\class{CoordSyst}
\class{Position}
\class{_CObj}

No documentation for this class.

\end{classdesc*}

\begin{classdesc*}{Camera}
Inherits:
\class{_Camera}
\class{CoordSyst}
\class{Position}
\class{_CObj}

Camera

The Camera specifies from where the scene is viewed.

Attributes are (see also CoordSyst and Widget for inherited attributes):

 - front, back : objects whose distance from the camera is not between front and back
   are clipped. Front defaults to 0.1 and back to 100.0.
   If the back / front ratio is too big, you loose precision in the depth buffer.

 - fov : the field of vision (or FOV), in degrees. Default is 60.0.

 - to\_render : the world that is rendered by the camera. Default is None, which means
   the root scene (as returned by get\_root()).

 - left, top, width, height : the viewport rectangle, in pixel. Use it if you want to
   render only on a part of the screen. It defaults to the whole screen.

 - ortho : True for orthogonal rendering, instead of perspective. Default is false.

 - partial : XXX ???. probably DEPRECATED by NoBackgroundAtmosphere.

\end{classdesc*}

\begin{classdesc*}{CellShadingShape}
Inherits:
\class{Shape}
\class{SavedInAPath}
\class{_Shape}
\class{_CObj}
\class{_CellShadingShape}
\class{_SimpleShape}
\class{_Shape}
\class{_CObj}

CellShadingShape

A Shape that use cell-shading for rendering.

\begin{datadesc}{DIRNAME}
\end{datadesc}
\end{classdesc*}

\begin{classdesc*}{CellShadingShapifier}
Inherits:
\class{SimpleShapifier}
\class{Shapifier}
\class{_CObj}

CellShadingShapifier

Shapifier for cell-shaded Shape.

Attributes are :

 - shadow (default 0) : enable shadows

 - shader : the material used for cell-shading lighting

 - outline\_color (default black) : the color of the outline

 - outline\_width (default 4.0) : the maximum line width when the cell-shaded shape
   is very near the camera (set to 0.0 to disable outlines)

 - outline\_attenuation (default : 0.3) : specify how much the distance affect the
   outline\_width

\end{classdesc*}

\begin{classdesc*}{Context}
Inherits:

No documentation for this class.

\end{classdesc*}

\begin{classdesc*}{CoordSyst}
Inherits:
\class{Position}
\class{_CObj}

No documentation for this class.

\begin{methoddesc}{add_mul_vector}{k, vector}
Translates a Position IN PLACE, by K * VECTOR.
Coordinates system conversion is performed if needed (=if the Position and
VECTOR are not defined in the same coordinates system).

For Vector, add\_mul\_vector means vectorial addition (translating a vector does
nothing !).
\end{methoddesc}

\begin{methoddesc}{add_vector}{vector}
Translates a Position IN PLACE.
Coordinates system conversion is performed if needed (=if the Position and
VECTOR are not defined in the same coordinates system).

For Vector, add\_vector means vectorial addition (translating a vector does
nothing !).
\end{methoddesc}

\begin{methoddesc}{add_xyz}{x, y, z}
Translates a Position by X, Y and Z.
\end{methoddesc}

\begin{methoddesc}{added_into}{new_parent}
Called when the CoordSyst is added into NEW\_PARENT, or removed from its previous parent
(in this case, NEW\_PARENT is None).
\end{methoddesc}

\begin{methoddesc}{advance_time}{proportion}
Called (by the Idler) when a piece of a round is achieved; default implementation does nothing.
PROPORTION is the proportion of the current round's time that has passed (1.0 for an entire round).
\end{methoddesc}

\begin{methoddesc}{begin_round}{}
Called (by the Idler) when a new round begins; default implementation does nothing.
\end{methoddesc}

\begin{methoddesc}{distance_to}{other}
Gets the distance between a Position and anOTHER.
\end{methoddesc}

\begin{methoddesc}{end_round}{}
Called (by the Idler) when a round is finished; default implementation does nothing.
\end{methoddesc}

\begin{methoddesc}{get_box}{}
Returns a box wrapping the given CoordSyst. The box is defined by 2 Point corresponding
to 2 opposite corners of the box, the box being aligned on the X, Y and Z axis.
\end{methoddesc}

\begin{methoddesc}{get_dimension}{}
Returns the dimension of a CoordSyst: a (width, height, depth) tuple.
For a World, this includes also all items inside the World.
\end{methoddesc}

\begin{methoddesc}{get_root}{}
Returns the "root parent" of a coordsyst, i.e. the scene (= the parent world that has no parent).
\end{methoddesc}

\begin{methoddesc}{get_sphere}{}
Returns a sphere wrapping the given CoordSyst. The sphere is defined by the center
and the radius (a Point and a float value).
\end{methoddesc}

\begin{methoddesc}{is_inside}{COORDSYST}
Returns true if self is inside COORDSYST, i.e. if COORDSYST is self, or self.parent, or
self.parent.parent or...
\end{methoddesc}

\begin{methoddesc}{look_at}{target}
Rotate so that this CoordSyst's *negative*-Z points at target.
\end{methoddesc}

\begin{methoddesc}{look_at_x}{target}
Rotate so that this object's positive-X points at target.
\end{methoddesc}

\begin{methoddesc}{look_at_y}{target}
Rotate so that this object's positive-Y points at target.
\end{methoddesc}

\begin{methoddesc}{move}{position}
Moves a Position to POSITION.
Coordinates system conversion is performed if needed (=if the Position and
POSITION are not defined in the same coordinates system).
\end{methoddesc}

\begin{methoddesc}{position}{}
Returns the position (a Point) at the same position than the Position.
\end{methoddesc}

\begin{methoddesc}{rotate}{ANGLE, A, B}
Rotate a CoordSyst about an axis, of ANGLE degrees.
The axis is defined by a Position A, and another Position or a Vector B.
\end{methoddesc}

\begin{methoddesc}{rotate_axe}{ANGLE, AXE}
Rotate a CoordSyst about an axis, of ANGLE degrees.
The axis is defined by a Vector AXE, and pass through the origin (0, 0, 0).
\end{methoddesc}

\begin{methoddesc}{rotate_axe_xyz}{ANGLE, X, Y, Z}
Rotate a CoordSyst about an (X, Y, Z) axis, of ANGLE degrees.
\end{methoddesc}

\begin{methoddesc}{rotate_x}{angle}
Rotate about the *parent's* Y axis, in degrees.
e.g., If you are facing positive-Y with head to positive-Z, it
spins like a car's steering wheel.
\end{methoddesc}

\begin{methoddesc}{rotate_xyz}{ANGLE, X1, Y1, Z1, X2, Y2, Z2}
Rotate a CoordSyst about an axis, of ANGLE degrees.
The axis is defined by a two point (X1, Y1, Z1) and (X2, Y2, Z2).
\end{methoddesc}

\begin{methoddesc}{rotate_y}{angle}
Rotate about the *parent's* X axis, in degrees.
e.g., If you are facing positive-Y with head to positive-Z, it
spins like a car's tire.
\end{methoddesc}

\begin{methoddesc}{rotate_z}{angle}
Rotate about the *parent's* Z axis, in degrees.
e.g., If you are facing positive-Y with head to positive-Z, it
spins like a spinning top.
\end{methoddesc}

\begin{methoddesc}{scale}{x, y, z}
Scales a CoordSyst by X, Y and Z (Changes its dimensions).
Negative values are accepted.
\end{methoddesc}

\begin{methoddesc}{set_dimension}{width, height, depth}
Sets the dimension of a CoordSyst ; in other worlds, scale the coordsyst so as it has
the given dimensions.
\end{methoddesc}

\begin{methoddesc}{set_identity}{}
Resets a CoordSyst (moves it to 0,0,0 and removes any rotation or scaling).
\end{methoddesc}

\begin{methoddesc}{set_scale_factors}{SCALE_X, SCALE_Y, SCALE_Z}
Sets the scale factors in the X, Y and Z dimension.
\end{methoddesc}

\begin{methoddesc}{set_xyz}{x, y, z}
Moves a CoordSyst to X, Y and Z.
\end{methoddesc}

\begin{methoddesc}{shift}{x, y, z}
Translates a CoordSyst by X, Y and Z, given in CoordSyst space.
\end{methoddesc}

\begin{methoddesc}{transform}{...}
DEPRECATED
\end{methoddesc}

\begin{methoddesc}{transform_point}{...}
DEPRECATED
\end{methoddesc}

\begin{methoddesc}{transform_vector}{...}
DEPRECATED
\end{methoddesc}

\begin{methoddesc}{turn_x}{angle}
Rotate about the *local* Y axis, in degrees.
\end{methoddesc}

\begin{methoddesc}{turn_y}{angle}
Rotate about the *local* X axis, in degrees.
\end{methoddesc}

\begin{methoddesc}{turn_z}{angle}
Rotate about the *local* Z axis, in degrees.
\end{methoddesc}

\begin{methoddesc}{vector_to}{other}
Gets the vector that starts at a Position and ends at OTHER.
\end{methoddesc}

\end{classdesc*}

\begin{classdesc*}{CylinderSprite}
Inherits:
\class{_CylinderSprite}
\class{_Sprite}
\class{CoordSyst}
\class{Position}
\class{_CObj}
\class{Sprite}
\class{_Sprite}
\class{CoordSyst}
\class{Position}
\class{_CObj}

CylinderSprite

A special kind of Sprite, that points toward the camera only in X and Z dimension, but
not Y. This is usefull e.g. for lightening spell effects, for which using a normal Sprite
would give a poor rendering if seen from top.

\end{classdesc*}

\begin{classdesc*}{Face}
Inherits:
\class{_Face}
\class{CoordSyst}
\class{Position}
\class{_CObj}

Face

A Face displays a polygon composed of several Vertices (see the Vertex class).
Notice that Face are SLOW ; Faces are normally used for building model but not for
rendering them. To get a fast rendering, you should put several Faces in a World, and
then compile the World into a Shape (see the modeling-X.py tutorial series).

According to the number of Vertices, the result differs:
 - 1 Vertex => Plot
 - 2        => Line
 - 3        => Triangle
 - 4        => Quad

All the vertices are expected to be coplannar.

Interesting properties are:

 - vertices: the list of Vertices

 - material: the material used to draw the face's surface

 - double\_sided: true if you want to see both sides of the Face. Default is false.

 - solid: true to enable the use of this Face for raypicking. Default is true.

 - lit: true to enable lighting on the Face. Default is true.

The following options are used when compiling the Face into a Shape,
but does not affect the rendering of the Face itself:

 - static\_lit: true to enable static lighting (faster). If true, when compiling the Face
   into a Shape, all Lights available will be applied as static lighting. Default is true.

 - smooth\_lit: true to compute per-vertex normal vectors, instead of per-face normal vector.
   This makes the Shape looking smooth (see tutorial modeling-smoothlit-1.py).
   Notice that Soya automatically disable smooth\_lit between 2 faces that makes a sharp
   angle (see Shapifier.max\_face\_angle attribute).
   Default is false.

\end{classdesc*}

\begin{classdesc*}{FixTraveling}
Inherits:
\class{_FixTraveling}
\class{Traveling}
\class{_CObj}

No documentation for this class.

\end{classdesc*}

\begin{classdesc*}{FlagFirework}
Inherits:
\class{_Particles}
\class{CoordSyst}
\class{Position}
\class{_CObj}

No documentation for this class.

\begin{methoddesc}{generate}{...}
\end{methoddesc}

\begin{methoddesc}{mygenerate}{...}
\end{methoddesc}

\begin{methoddesc}{regenerate}{...}
\end{methoddesc}

\begin{methoddesc}{remove}{...}
\end{methoddesc}

\begin{methoddesc}{subgenerate}{...}
\end{methoddesc}

\end{classdesc*}

\begin{classdesc*}{FlagSubFire}
Inherits:
\class{_Particles}
\class{CoordSyst}
\class{Position}
\class{_CObj}

No documentation for this class.

\begin{methoddesc}{generate}{...}
\end{methoddesc}

\end{classdesc*}

\begin{classdesc*}{Font}
Inherits:

No documentation for this class.

\begin{methoddesc}{create_glyphs}{...}
\end{methoddesc}

\begin{methoddesc}{draw}{...}
\end{methoddesc}

\begin{methoddesc}{draw_area}{...}
\end{methoddesc}

\begin{methoddesc}{get_glyph}{...}
\end{methoddesc}

\begin{methoddesc}{get_print_size}{...}
\end{methoddesc}

\begin{methoddesc}{wordwrap}{...}
return a tuple like:
( max\_line\_width, total\_height, wrapped text )
\end{methoddesc}

\begin{methoddesc}{wordwrap_line}{...}
return a tuple like :
( max\_line\_width, total\_height, [ ('line1', line1width), ('line2', linewidth2) ... ] )
\end{methoddesc}

\end{classdesc*}

\begin{classdesc*}{Fountain}
Inherits:
\class{_Particles}
\class{CoordSyst}
\class{Position}
\class{_CObj}

No documentation for this class.

\begin{methoddesc}{generate}{...}
\end{methoddesc}

\end{classdesc*}

\begin{excdesc}{GLError}
Inherits:
\class{StandardError}
\class{Exception}

No documentation for this class.

\end{excdesc}

\begin{classdesc*}{Glyph}
Inherits:

No documentation for this class.

\end{classdesc*}

\begin{classdesc*}{Idler}
Inherits:

Idler

A main loop, with FPS regulation.

Interesting attributes:

 - fps (read only): the frame rate (number of frame per second, a usefull speed indicator).

 - running (read only): true if the Idler is idling (=running).

 - next\_round\_tasks: a list of callable (taking no arg) that will be called once, just
   after the beginning of the next round.

 - scenes: the scenes associated to this idler.

 - round\_duration: The duration of a round. Round is the idler's time unit. The idler calls
   successively begin\_round(), advance\_time() (possibly several times) and end\_round(); it
   is granted that ALL rounds correspond to a period of duration ROUND\_DURATION (though
   the different period may not be regularly spread over time).
   Default is 0.030.

 - min\_frame\_duration: minimum duration for a frame. This attribute can be used to limit
   the maximum FPS to save CPU time; e.g. FPS higher than 30-40 is usually useless.
   Default is 0.020, which limits FPS to 40 in theory and to about 33 in practice
   (I don't know why there is a difference between theory and practice !).

\begin{methoddesc}{advance_time}{proportion}
Called by Idler.idle when a piece of a round has occured; default implementation delegates to Idler.scene.advance\_time.
PROPORTION is the proportion of the current round's time that has passed (1.0 for an entire round).
\end{methoddesc}

\begin{methoddesc}{begin_round}{}
Called by Idler.idle when a new round begins; default implementation delegates to Idler.scene.begin\_round.
\end{methoddesc}

\begin{methoddesc}{end_round}{}
Called by Idler.idle when a round is finished; default implementation delegates to Idler.scene.end\_round.
\end{methoddesc}

\begin{methoddesc}{idle}{}
Starts idling with the current thread. This method never finishes, until you call Idler.stop().
\end{methoddesc}

\begin{methoddesc}{render}{}
Called by Idler.idle when rendering is needed; default implementation calls soya.render.
\end{methoddesc}

\begin{methoddesc}{reset}{...}
\end{methoddesc}

\begin{methoddesc}{start}{}
Starts idling with a new thread (see Idler.idle() if you don't want to create a new thread).
\end{methoddesc}

\begin{methoddesc}{stop}{VALUE = None}
Stops idling. The stopping may not occur immediately, but at the end of the next iteration.
Idler.stop() causes Idler.idle() to returns ; VALUE is the (optionnal) value that Idler.idle() will return.
\end{methoddesc}

\end{classdesc*}

\begin{classdesc*}{Image}
Inherits:
\class{SavedInAPath}
\class{_Image}
\class{_CObj}

A Soya image, suitable for e.g. texturing.

Attributes are:

 - pixels : the raw image data (e.g. in a form suitable for PIL).

 - width.

 - height.

 - nb\_color: the number of color channels (1 => monochrome, 3 => RGB, 4 => RGBA).

\begin{datadesc}{DIRNAME}
\end{datadesc}
\begin{methoddesc}{save}{klass, filename=None}
\end{methoddesc}

\end{classdesc*}

\begin{classdesc*}{Land}
Inherits:
\class{_Land}
\class{CoordSyst}
\class{Position}
\class{_CObj}

No documentation for this class.

\end{classdesc*}

\begin{classdesc*}{Light}
Inherits:
\class{_Light}
\class{CoordSyst}
\class{Position}
\class{_CObj}

Light

A Light is a 3D object that enlights the other objects.

Attributes are (see also CoordSyst for inherited attributes):

 - directional : True for a directional light (e.g. like the sun), instead of a
   positional light. The position of a directional light doesn't matter, and only
   the constant component of the attenuation is used. Default is false.
 - constant, linear and quadratic : the 3 components of the light attenuation. Constant
   reduces the light independently of the distance, linear increase with the distance,
   and quadratic increase the squared distance. Default is 1.0, 0.0 and 0.0.
 - cast\_shadow : True if the light cast shadows on Shapes that have shadows enabled.
   Default is true.
 - shadow\_color : the color of the shadow . Default is a semi-transparent black
   (0.0, 0.0, 0.0, 0.5).
 - top\_level : XXX ???
 - static : True if the light can be used for static lighting when compiling a World into
   a Shape. Default is true.
 - ambient : the light's ambient color, which is not affected by the light's orientation
   or attenuation. Default is black (no ambient).
 - diffuse : the light's color. Default is white.
 - specular : the light's specular color. Default is white.

\end{classdesc*}

\begin{classdesc*}{Material}
Inherits:
\class{SavedInAPath}
\class{_Material}
\class{_CObj}

Material

A material regroups all the surface attributes, like colors, shininess and
texture. You should NEVER use None as a material, use soya.\_DEFAULT\_MATERIAL instead.

Attributes are:

- diffuse: the diffuse color.

- specular: the specular color; used for shiny part of the surface.

- emissive: the emissive color; this color is applied even in the dark.

- separate\_specular: set it to true to enable separate specular; this usually
  results in a more shiny specular effect.

- shininess: the shininess ranges from 0.0 to 128.0; 0.0 is the most metallic / shiny
  and 128.0 is the most plastic.

- texture: the texture (a soya.Image object, or None if no texture).

- clamp: set it to true if you don't want to repeat the texture when the texture
  coordinates are out of the range 0 - 1.0.

- additive\_blending: set it to true for additive blending. For semi-transparent surfaces
  (=alpha blended) only. Usefull for special effect (Ray, Sprite,...).

\begin{datadesc}{DIRNAME}
\end{datadesc}
\end{classdesc*}

\begin{classdesc*}{NoBackgroundAtmosphere}
Inherits:
\class{_NoBackgroundAtmosphere}
\class{_Atmosphere}
\class{_CObj}
\class{Atmosphere}
\class{_Atmosphere}
\class{_CObj}

NoBackgroundAtmosphere

An Atmosphere with no background. It is usefull is you want to render a 3D scene over
another 3D scene.

\end{classdesc*}

\begin{classdesc*}{Particles}
Inherits:
\class{_Particles}
\class{CoordSyst}
\class{Position}
\class{_CObj}

No documentation for this class.

\end{classdesc*}

\begin{classdesc*}{Point}
Inherits:
\class{_Point}
\class{Position}
\class{_CObj}

A Point is just a 3D position. It is used for math computation, but it DOESN'T display
anything.

\end{classdesc*}

\begin{classdesc*}{Portal}
Inherits:
\class{_Portal}
\class{CoordSyst}
\class{Position}
\class{_CObj}

No documentation for this class.

\begin{methoddesc}{pass_through}{self, coordsyst}
Portal.pass\_though(self, coordsyst)

Makes COORDSYST pass through the portal. If needed (=if coordsyst.parent is self.parent),
it removes COORDSYST from its current parent and add it in the new one,
at the right location.
If coordsyst is a camera, it change the 'to\_render' attribute too.

The passing through does NOT occur immediately, but after the beginning of the round
(this is usually what you need, in order to avoid that moving COORDSYST from a parent
to another makes it plays twice).
\end{methoddesc}

\end{classdesc*}

\begin{classdesc*}{Position}
Inherits:
\class{_CObj}

A 3D position.

In Soya 3D, a position is defined by 3 coordinates (x, y, z) AND the
coordinates system in which the coordinates are defined (sometime
called the "parent").

Points and Vectors are used for math computation; my own experiences in 3D
have convinced me that ANY complicated 3D computation can be heavily
simplified by the use of coordinates system conversions. As Soya 3D
associates the coordinates system along with the coordinates values, it
performs coordinates system conversion automagically !

\begin{methoddesc}{position}{}
Returns the position of this Position (a Point or a Vector, whatever Position is).
\end{methoddesc}

\end{classdesc*}

\begin{classdesc*}{PythonCoordSyst}
Inherits:
\class{CoordSyst}
\class{Position}
\class{_CObj}

A CoordSyst whose rendering part is implemented in Python.
This class is destinated to be inherited.
You should override the batch and render methods.

\begin{methoddesc}{batch}{}
Should return a tuple (TYPE, COORDSYST, MATERIAL), where type can be :
  0 if invisible / not drawn
  1 if drawn without alpha
  2 if drawn with alpha
COORDSYST is the coordinate system to use for rendering the object (usually self).
\end{methoddesc}

\begin{methoddesc}{render}{}
Should perform the rendering (e.g. by calling OpenGL operation).
\end{methoddesc}

\begin{methoddesc}{shadow}{...}
\end{methoddesc}

\end{classdesc*}

\begin{classdesc*}{RaypickContext}
Inherits:

No documentation for this class.

\begin{methoddesc}{raypick}{origin, direction, distance = -1.0, half_line = 1, cull_face = 1, p = None, v = None}
See World.raypick
\end{methoddesc}

\begin{methoddesc}{raypick_b}{origin, direction, distance = -1.0, half_line = 1, cull_face = 1, p = None, v = None}
See World.raypick\_b
\end{methoddesc}

\end{classdesc*}

\begin{classdesc*}{RaypickData}
Inherits:

No documentation for this class.

\end{classdesc*}

\begin{classdesc*}{Renderer}
Inherits:

No documentation for this class.

\end{classdesc*}

\begin{classdesc*}{SavedInAPath}
Inherits:

SavedInAPath

Base class for all objects that can be saved in a path, such as Material,
World,...

\begin{datadesc}{DIRNAME}
\end{datadesc}
\begin{methoddesc}{get_filename}{self}
\end{methoddesc}

\begin{methoddesc}{loaded}{self}
\end{methoddesc}

\begin{methoddesc}{save}{self, filename=None}
SavedInAPath.save(filename = None)

Saves this object. If no FILENAME is given, the object is saved in the path,
using its filename attribute. If FILENAME is given, it is saved at this
location.
\end{methoddesc}

\begin{methoddesc}{set_filename}{self, filename}
\end{methoddesc}

\end{classdesc*}

\begin{classdesc*}{Shape}
Inherits:
\class{SavedInAPath}
\class{_Shape}
\class{_CObj}

Shape

A Shape is an optimized model. Shapes cannot be modified, but they are rendered very
quickly, and they can be used several time, e.g. if you want to 2 same cubes in a scene.
Shapes are used in conjunction with Volume.

\begin{datadesc}{DIRNAME}
\end{datadesc}
\end{classdesc*}

\begin{classdesc*}{Shapifier}
Inherits:
\class{_CObj}

No documentation for this class.

\end{classdesc*}

\begin{classdesc*}{SimpleShape}
Inherits:
\class{Shape}
\class{SavedInAPath}
\class{_Shape}
\class{_CObj}
\class{_SimpleShape}
\class{_Shape}
\class{_CObj}

SimpleShape

The most basic class of Shape.

\begin{datadesc}{DIRNAME}
\end{datadesc}
\end{classdesc*}

\begin{classdesc*}{SimpleShapifier}
Inherits:
\class{Shapifier}
\class{_CObj}

SimpleShapifier

Shapifier for simple / normal / regular Shape. The SimpleShapifier attributes allows to
customize the World -> Shape computation.

Attributes are :

 - shadow (default 0) : enable shadows

 - max\_face\_angle (default 80.0) : if the angle (in degree) between 2 faces is less than
   this value, vertices of the two faces can be merged if they are enough close. Set it
   to 180.0 or more to disable this feature.

\end{classdesc*}

\begin{classdesc*}{SkyAtmosphere}
Inherits:
\class{_SkyAtmosphere}
\class{_Atmosphere}
\class{_CObj}
\class{Atmosphere}
\class{_Atmosphere}
\class{_CObj}

SkyAtmosphere

An Atmosphere with a skybox and/or a sky/cloud effect.

In addition to those of Atmosphere, attributes are :

 - sky\_box: the sky box, a tuple of 5 or 6 materials that are displayed on the 6 faces of
   the sky box (which is a cube). Use an empty tuple () to disable the sky box (default value).

 - sky\_color: the color of the sky (an (R, G, B, A) tuple of four floats).
   Use a sky\_color with an alpha component of 0.0 to disable cloud / sky effet.

 - cloud: the cloud material, which is used to add a cloud-like effect to the sky.
   Coulds move according to the camera.

\end{classdesc*}

\begin{classdesc*}{Smoke}
Inherits:
\class{_Particles}
\class{CoordSyst}
\class{Position}
\class{_CObj}

No documentation for this class.

\begin{methoddesc}{generate}{...}
\end{methoddesc}

\end{classdesc*}

\begin{classdesc*}{Sprite}
Inherits:
\class{_Sprite}
\class{CoordSyst}
\class{Position}
\class{_CObj}

Sprite

A 2D sprite, displayed as a 3D (e.g. think to very old 3D games).
Today, sprite are usefull for special effect like explosion, or halo.
The Sprite 2D texture always points toward the camera.

Attributes are :

 - material: the material of the Sprite, including the texture.

 - color: the color of the Sprite (defaults to white).

 - width and height: the size of the Sprite, in 3D coordinates values (not pixel values).
   Both defaults to 0.5.

 - lit: if true (default), lighting affects the sprite, and if false, it doesn't.

\end{classdesc*}

\begin{classdesc*}{ThirdPersonTraveling}
Inherits:
\class{_ThirdPersonTraveling}
\class{Traveling}
\class{_CObj}

No documentation for this class.

\end{classdesc*}

\begin{classdesc*}{Traveling}
Inherits:
\class{_CObj}

No documentation for this class.

\begin{methoddesc}{best_direction}{...}
\end{methoddesc}

\begin{methoddesc}{best_position}{...}
\end{methoddesc}

\end{classdesc*}

\begin{classdesc*}{TravelingCamera}
Inherits:
\class{_TravelingCamera}
\class{_Camera}
\class{CoordSyst}
\class{Position}
\class{_CObj}

No documentation for this class.

\end{classdesc*}

\begin{classdesc*}{TreeShape}
Inherits:
\class{Shape}
\class{SavedInAPath}
\class{_Shape}
\class{_CObj}
\class{_TreeShape}
\class{_SimpleShape}
\class{_Shape}
\class{_CObj}

TreeShape

A Shape that use a BSP-like tree to optimize rendering and raypicking.

\begin{datadesc}{DIRNAME}
\end{datadesc}
\end{classdesc*}

\begin{classdesc*}{TreeShapifier}
Inherits:
\class{SimpleShapifier}
\class{Shapifier}
\class{_CObj}

TreeShapifier

Shapifier for tree-based Shape. Yields a TreeShape instead of a SimpleShape.
TreeShape are optimized for big shape with lots of faces, espescially if all the faces
are not visible at the same time (e.g. a game level). Both rendering and raypicking
are optimized.
Internally, the shape is broken down into several hierarchinal nodes, each node grouping
close faces.

Attributes are :

 - shadow (default 0) : NOT IMPLEMENTED YET for tree

 - collapsing\_distance (default 0.9) : this parameter tunes how many nodes are created.
   If a child node's radius > parent node's radius X collapsing\_distance, the child
   and parent nodes are merged.

 - quality (default 0) : set to 1 to compute a slower but slightly more performant tree.

 - max\_child\_radius (default 0.5) : the maximum children node's radius, expressed in ratio
   of the parent node's radius. Meaningfull only when quality == 0

\end{classdesc*}

\begin{classdesc*}{Vector}
Inherits:
\class{_Vector}
\class{_Point}
\class{Position}
\class{_CObj}

A Vector is a 3D vector (and not a kind of list or sequence ;-). Vectors are
useful for 3D math computation. 

Most of the math operators, such as +, -, *, /, abs,... work on Vectors and do
what they are intended to do ;-)

\end{classdesc*}

\begin{classdesc*}{Vertex}
Inherits:
\class{_Vertex}
\class{_Point}
\class{Position}
\class{_CObj}

Vertex

A Vertex is a subclass of Point, which is used for building Faces.
A Vertex doesn't display anything ; you MUST put it inside a Face.

Attributes are (see also Point for inherited attributes):

 - diffuse: the vertex diffuse color (also aliased to 'color' for compatibility)
 - emissive: the vertex emissive color (lighting-independant color)
 - tex\_x, tex\_y: the text
ure coordinates (sometime called U and V).

\end{classdesc*}

\begin{classdesc*}{Volume}
Inherits:
\class{_Volume}
\class{CoordSyst}
\class{Position}
\class{_CObj}

Volume

A Volume is a Soya 3D object that display a Shape. The Volume contains data about the
position, the orientation and the scaling, and the Shape contains the geometric data.

This separation allows to use several time the same Shape at different position, without
dupplicating the geometric data.

Attributes are (see also CoordSyst for inherited attributes):

 - shape : the Shape (a Shape object, defaults to None).

\end{classdesc*}

\begin{classdesc*}{WaterCube}
Inherits:
\class{_WaterCube}
\class{CoordSyst}
\class{Position}
\class{_CObj}

No documentation for this class.

\end{classdesc*}

\begin{classdesc*}{World}
Inherits:
\class{SavedInAPath}
\class{_World}
\class{_Volume}
\class{CoordSyst}
\class{Position}
\class{_CObj}
\class{Volume}
\class{_Volume}
\class{CoordSyst}
\class{Position}
\class{_CObj}

World

A World is a Soya 3D object that can contain other Soya 3D objects, including other Worlds.
Worlds are used to group 3D objects ; when a World is moved, all the objects it contains
are moved too, since they are part of the World.
Mostly for historical reasons, World is a subclass of Volume, and thus can display a Shape.

Worlds can be saved in the "worlds" directory ; see SavedInAPath.

Attributes are (see also Volume, CoordSyst and SavedInAPath for inherited attributes):

 - children : the list of 3D object contained in the World (default to an empty list).
   use World.add(coordsyst) and World.remove(coordsyst) for additions and removals.

 - atmosphere : the atmosphere specifies atmospheric properties of the World (see
   Atmosphere). Default is None.

 - shapifier : the shapifier specifies how the World is compiled into Shape.
   Default is None, which result in the use of the default Shapifier, which is
   SimpleShapifier().

\begin{datadesc}{DIRNAME}
\end{datadesc}
\end{classdesc*}

\subsection{Constants}
\begin{datadesc}{ALL_EVENTS}
\end{datadesc}
\begin{datadesc}{BLACK}
\end{datadesc}
\begin{datadesc}{DATADIR}
\end{datadesc}
\begin{datadesc}{DEFAULT_MATERIAL}
\end{datadesc}
\begin{datadesc}{IDLER}
\end{datadesc}
\begin{datadesc}{PARTICLE_DEFAULT_MATERIAL}
\end{datadesc}
\begin{datadesc}{SDL_ALLEVENTS}
\end{datadesc}
\begin{datadesc}{SHADER_DEFAULT_MATERIAL}
\end{datadesc}
\begin{datadesc}{TRANSPARENT}
\end{datadesc}
\begin{datadesc}{VERSION}
\end{datadesc}
\begin{datadesc}{WHITE}
\end{datadesc}
