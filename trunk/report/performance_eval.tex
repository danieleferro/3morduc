\section{Performance Evaluation}
\label{sec:performance_evaluation}

This chapter will describe some tests performed at the University of Hertfordshire
(Hatfield, UK), along with their results. The R.E.A.R. framework has been tested with
the "sweep metric algorithm" (chapter \ref{subsec:sweep_metric_algorithm}), which can
be configured with some initial parameters (chapter \ref{subsec:parameters}).
%

%
The final target is to underline the chief differences occurring when one or more parameters
change their values, within defined ranges. After asking users their options and perceptions
about teleguiding the robot with an exocentric vision, we could at the end identify the advantages
and the disadvantages correlated with different sets of initial parameters.
%

%
Data about robot status and egocentric images are not retrieved dynamically from an active server (the
robot itself or a simulator), but read from files formerly stored. This means that user can not choose
the direction in which the robot moves, but simply request the next step. All the step form a path
already performed by the robot and saved in log files. 

%

%
Be careful that this tests do not have a scientific validity. In literature there is a lot of theories
and rules to follow in order to demonstrate or confute a specific hypothesis. In our case, we simply
intend to test R.E.A.R. application by presenting it to users who were not involved in its development,
noting down their impressions. Even tough, probably, the basic rules for testing has not been followed,
we think our test case can be useful to identify future improvements (see chapter \ref{sec:future_works}).

\subsection{Parameters}
\label{subsec:parameters}

Testing the "sweep metric algorithm" (see chapter \ref{subsec:sweep_metric_algorithm}) means to define
the parameters hold by the \texttt{SweepMetricCalc} objects. All these parameters are set with the
constructor method, whose signature is the following:
\\
%
\begin{verbatim}
SweepMetricCalc::SweepMetricCalc( float sweep_angle,
				  float angle_offset,
				  float mu_distance,
				  float sigma_distance,
				  float mu_angle,
				  float sigma_angle
				  )
\end{verbatim}
%

With six parameters it is tricky to evaluate how changing a single one affects on all the others. For this
reason we decided to fix same values when testing.
%

%
Before describing which parameters have been chosen to be fix and which to be mobile, an in depth explanation
about their meaning is needed. For a better comprehension see chapter \ref{subsec:sweep_metric_algorithm}; all
the angles are specified in degrees.

%
\begin{itemize}

  \item \texttt{sweep\_angle} \\
    half the angle which defines the "sweep area" 
  \item \texttt{angle\_offset} \\
    maximum difference allowed between robot and image orientation 

  \item \texttt{mu\_distance} \\
    expected value (i.e. mean value) for the Gaussian which assigns the score on the basis of distance between
    image and robot
  \item \texttt{sigma\_distance} \\
    standard deviation for the Gaussian which assigns the score on the basis of distance between image and robot

  \item \texttt{mu\_angle} \\
    expected value (i.e. mean value) for the Gaussian which assigns the score on the basis of orientation difference
    between image and robot
  \item \texttt{sigma\_angle} \\
    standard deviation for the Gaussian which assigns the score on the basis of orientation difference between image
    and robot

\end{itemize}
%

To begin with, both the standard deviation \texttt{mu\_distance} and \texttt{mu\_angle} belongs to the set of fixed
values. Changing the standard deviation in a Gaussian function means to 
