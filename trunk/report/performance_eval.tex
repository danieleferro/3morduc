\section{Performance Evaluation}
\label{sec:performance_evaluation}

This chapter will describe some tests performed at the University of Hertfordshire
(Hatfield, UK), along with their results. 
%
The application described in \ref{sec:concr} has been tested 
featuring the \textit{sweep metric algorithm} (see section 
\ref{subsec:sweep_metric_algorithm}), which was 
given different paramaters per each testing session.
%

%
The test target is to underline the main differences 
occurring when one or more parameters change their values, 
within defined ranges. After asking users their options and perceptions
about teleguiding the robot with an exocentric vision, we could at the end identify the advantages
and the disadvantages correlated with different sets of initial parameters.
%

%
Since working offline (see section \ref{sec:concr}), users can not actually 
command the robot, but simply request the application to go one step
further and, hence, just see the robot performing the trajectory 
it did during the recorded test session.
%

%
Be advised that this tests do not have a scientific validity. In literature there is a lot of theories
and rules to follow in order to demonstrate or confute a specific hypothesis. In our case, we simply
intend to test R.E.A.R. application by presenting it to users who were not involved in its development,
noting down their impressions. Even tough, probably, the basic rules for testing has not been followed,
we think our test case can be useful to identify future improvements (see chapter \ref{sec:future_works}).

\subsection{Parameters}
\label{subsec:parameters}

Testing the "sweep metric algorithm" (see chapter \ref{subsec:sweep_metric_algorithm}) means to define
the parameters hold by the \texttt{SweepMetricCalc} objects. All these parameters are 
to be passed to the class constructor, whose signature is the following:
\\
%
\begin{verbatim}
SweepMetricCalc::SweepMetricCalc( float sweep_angle,
				  float angle_offset,
				  float mu_distance,
				  float sigma_distance,
				  float mu_angle,
				  float sigma_angle
				  )
\end{verbatim}
%

With six parameters it is tricky to evaluate how changing a single one affects on all the others. For this
reason we decided to fix same values when testing.
%

%
Before describing which parameters have been chosen to be fixed and which to be variable, an in depth explanation
about their meaning is needed. For a better comprehension see chapter \ref{subsec:sweep_metric_algorithm}; all
the angles are specified in degrees.

%
\begin{itemize}

  \item \texttt{sweep\_angle} \\
    half the angle which defines the "sweep area" 
  \item \texttt{angle\_offset} \\
    maximum difference allowed between robot and image orientation 

  \item \texttt{mu\_distance} \\
    expected value (i.e. mean value) for the Gaussian which assigns the score on the basis of distance between
    image and robot
  \item \texttt{sigma\_distance} \\
    standard deviation for the Gaussian which assigns the score on the basis of distance between image and robot

  \item \texttt{mu\_angle} \\
    expected value (i.e. mean value) for the Gaussian which assigns the score on the basis of orientation difference
    between image and robot
  \item \texttt{sigma\_angle} \\
    standard deviation for the Gaussian which assigns the score on the basis of orientation difference between image
    and robot

\end{itemize}
%

%
To begin with, \texttt{sweep\_angle} has been considered a fixed parameter during the tests. We set it to 45 degrees,
because greater values do not seem (in previous tests) to get any advantages. On the other hand, value less than 45
degrees would not include enough images to teleguide the robot properly.
%

%
\texttt{angle\_offset} is another fixed parameter, set to 40 degrees. Exceeding this value, we risk not to include
the robot within the camera field of view, when camera and robot present an orientation offset greater than 40 degrees.
Hence, all those images whose orientation exceedes 40 degrees cannot compete to be the background image, 
and have to be excluded.
%

%
The standard deviations \texttt{sigma\_distance} and \texttt{sigma\_angle} belong to the set of
fixed values too. Changing the standard deviation in a Gaussian function means only to increase or decrease its higher
point, without affecting other Gaussian properties. Because later on it will be selected the greatest value among
all the returned ones, without any absolute reference, we do not care about the standard deviations value.
\texttt{sigma\_distance} and \texttt{sigma\_angle} are set with a default positive amount.
%

%
In terms of Gaussian function, the \texttt{mu\_angle} represent the mean value and, at the same time, the point where
the Gaussian function is centred. By computing the function with \texttt{mu\_angle} in input, it will return the possible
maximum value. We remember that this function is used to calculate a score, which has to be as greater as the difference
between the robot and image orientations are equal. Because the function input is the difference between the two angles, it
follows that the returned value must be maximum when the input is zero, decreasing when the input moves away from zero.
%

%
\texttt{mu\_angle} is therefore a fixed parameter, set to zero.
%

%
At last, the \texttt{mu\_distance} is the unique variable parameter. All the general consideration made before for the 
\texttt{mu\_angle} remain still valid, but this time we want to obtain the maximum score when the difference between
robot and image position is equal to a determinant positive value, decreasing when the difference moves away from it.
%

%
If we choose a \texttt{mu\_distance} close to zero, the selected background image will be near to the robot actual position.
The robot will be drawn only partially and the exocentric vision will be similar to the egocentric. The more
\texttt{mu\_distance} moves away from zero (with a positive value), the more the application will tend to draw
(when possible) all the robot to provide a full exocentric control vision, because far image will gain an higher score.
%

%
Tests has been performed with three different paths, each one with three different \texttt{mu\_distance} values: 5, 15
and 25. The evidences deduced are examined in the next paragraph (\ref{subsec:testevaluation}).

\subsection{Test evaluation}
\label{subsec:testevaluation}

\subsubsection{Square test evaluation}
\label{subsubsec:squaretest}

\subsubsection{Ellipse test evaluation}
\label{subsubsec:ellipsetest}

\subsubsection{Zig-zag test evaluation}
\label{subsubsec:zigzagtest}

\subsubsection{Final considerations}
\label{subsubsec:finalconsiderations}
