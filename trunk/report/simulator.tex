\section{Setting up a 3morduc simulator}
\label{sec:simulator}

\newline In order to implement the exocentric vision \ref{sec:exo} we need a wide set of data provided by the robot.
These information consist of the camera images (the egocentric point of view) and of a simple textual file filled with 
the robot status. The latter let us know the robot position and its odometric data, because it would be impossible to know 
where exactly draw the robot without this awareness.
\newline The first examples of the exocentric vision need a large static space where the robot can move in, but the real robot 
(named 3morduc) can not be easily teleguided in wide environment because of its size and the lack of a large room in Catania. For 
these reasons the simulator seemed to be necessary.








%\cite{sugimoto}
