\setcounter{figure}{0}
\setcounter{table}{0}
\setcounter{lstlisting}{0}

\chapter{Future works}
\label{future_works}
\minitoc

This section contains some tips for the future developments of 
\framework{}.
\\
It would be good to make a comparison between the \textit{sweep angle
algorithm} and selection method number three described in \cite{sugimoto}.
\\
The formula used in \cite{sugimoto} to compare the view-point of an image with
the robot's current position and direction is not clear at all. Some parameters
are ambiguous and there is not an exhaustive explanation about the formula that
ties them together.
\\
After resolving and implementing the formula, it could be interesting to evaluate
the same case tests with the two approaches, in order to underline the advantages
and the disadvantages shown by each method and compare them.
\\
New implementations of the \texttt{IDataLogic} interface could be 
developed. One could, for instance, interact (by a socket or another 
data stream) with the simulator, making the whole system working \textit{online},
even though real \morduc{} robot is not powered on, connected or ready
to receive commands. Since client should communicate with a custom 
simulator, the communication protocol can be chosen by the developers.
\\
In document \cite{morduc:neri}, Neri and others prove (by several tests) that \textit{3D
vision guarantees a major precision in the teleguide and good performances on the obstacles
avoidance}. An interesting future development could add the stereoscopic vision to a concrete
\framework{} application, in order to merge the advantages brought by the two different
approaches in robot teleguiding.
\\
To implement 3D vision, the robot server must provide both the right and left camera images,
whereas the OpenGL functions must draw robot in the proper way on the images retrieved, to render
the 3D effect. More details depends on the 3D technologies chosen by the developer: shutter-glasses,
anaglyph, polarized, or other.
\\
The work presented in this document dis not take into account 
collisions. If the environment the robot moves in presents walls 
or obstacles to avoid, it could be useful to advise user in case 
of an imminent or already happened collision.
\\
A signalling system could be implemented once again with OpenGL, therefore 
with augmented reality. The laser value, provided by the \morduc{},
could help to understand when and where draw warnings, as shown in 
\cite{morduc:macalusodetommaso}.
\\
A more simple, but doubtless useful, future upgrade would consist in 
creating a graphical interface, which allows user to define all the initial
parameters (exposed in section \ref{sourcecode:downloadrun:run})
in a more friendly way. 
Currently, all these parameters (e.g. the number of log session or the 
optimal distance) are given by command line.
