\section{Setting up a texture}
\label{opengl:setting_texture}

According to \cite{opengl:distilled} \textit{texture mapping 
is a concept that takes a moment to grasp but a lifetime 
to master.}
\\
As for our work, we were simply interested in drawing an 
image as the background of a window. It is not a complex 
task but, it could be very tricky, especially when one 
does not have a deep understanding of what is happening 
\textit{under the hood}.
\\
OpenGL provides 1D, 2D and 3D textures. A 2D texture is enough for our
purposes.
\\
There exists a well-defined sequence of actions to draw a texture 
into an OpenGL application. Such a sequence made of the following steps:

\begin{enumerate}
\item obtain an unused texture object identifier with \texttt{glGenTextures()}, 
  and create a texture object using \texttt{glBindTexture()}
\item set texture-object state parameters
\item specify the texture image using \texttt{glTexImage2D()} 
  or \texttt{gluBuild2DMipmaps()}
\item before rendering geometry that uses the texture object, 
  bind the texture object with glBindTexture()
\item before rendering geometry, enable texture mapping
\item send geometry to OpenGL with appropriate texture 
  coordinates
\end{enumerate}

Let us show an example of how to take a preloaded image, binding it 
to a texture and draw it as the background of a window. First of all, 
we would like to define a helper structure which we will call \texttt{Image}
to store the actual image and its size.
\\
\begin{lstlisting}[caption={The Image structure}, label={code:image}]
struct Image {
  unsigned long sizeX;
  unsigned long sizeY;
  char * data;
};

typedef struct Image Image;
\end{lstlisting}

Now, let us suppose to have defined a \texttt{ImageLoad()} function 
that loads an image from disk and returns an object of class \texttt{Image}. 
Let us see how to put in code the procedure described above:
\\
\begin{lstlisting}[caption={Texture example}, label={code:texturemapping}]
  Gluint * texture;
  Image * image;
    
  // allocate space for the image
  image = (Image *) malloc(sizeof(Image));

  if (image == NULL) {
    // ERROR!
    exit(0);
  }

  // load image from disk
  if (!ImageLoad("image.bmp", image)) {
    exit(1);
  }        

  // obtain an unused texture object
  glGenTextures(1, texture);

  // Bind 2d texture (x and y size)
  glBindTexture(GL_TEXTURE_2D, * texture);   

  // now let us set up some state parameters:
  // scale linearly when image bigger than texture
  glTexParameteri(GL_TEXTURE_2D, GL_TEXTURE_MAG_FILTER,
                                 GL_LINEAR);
 
  // scale linearly when image smaller than texture
  glTexParameteri(GL_TEXTURE_2D, GL_TEXTURE_MIN_FILTER,
                                 GL_LINEAR); 

  gluBuild2DMipmaps(GL_TEXTURE_2D, 3, image1->sizeX, 
                    image1->sizeY, GL_RGB, GL_UNSIGNED_BYTE, 
                    image1->data);

  // enable texture mapping
  glEnable(GL_TEXTURE_2D);
  glMatrixMode(GL_PROJECTION);
  glPushMatrix();
  glLoadIdentity();
  glMatrixMode(GL_MODELVIEW);
  glPushMatrix();
  glLoadIdentity();

  // deactivate depth (Z Axis)
  glDepthMask(false);

  glBegin( GL_QUADS );

  // actually map texture
  {
    glTexCoord2f( 0.f, 0.f );
    glVertex2f( -1, -1 );

    glTexCoord2f( 0.f, 1.f );
    glVertex2f( -1, 1.f );

    glTexCoord2f( 1.f, 1.f );
    glVertex2f( 1.f, 1.f );

    glTexCoord2f( 1.f, 0.f );
    glVertex2f( 1.f, -1 );
  }

  glEnd();

  // reactivate depth (Z axis)
  glDepthMask(true);

  glPopMatrix();
  glMatrixMode(GL_PROJECTION);
  glPopMatrix();
  glMatrixMode(GL_MODELVIEW);
  glDisable(GL_TEXTURE_2D);
\end{lstlisting}
