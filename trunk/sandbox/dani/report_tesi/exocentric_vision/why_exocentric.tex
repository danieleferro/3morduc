\subsection{Why exocentric vision ?}
\label{sec:exo:why_exocetric}

Teleoperated mobile robots prove to be extremely useful 
when there is the need for performing operations in places that 
are inaccessible or dangerous for human beings - e.g. 
search-and-rescue missions within unknown regions or into 
collapsed buildings, caves, etc.
\\
The supervisors' research group has been widely involved 
during the last years in such a field: \cite{livatino2010}.
As testing platform, they have been using \textit{3morduc},
a differential-driven mobile robot - see chapter \ref{sec:3morduc} -
equipped with a pair of \textit{Videre Design} \cite{videredesign} 
stereo cameras and a laser scanner.
\\
They have been focused 
mainly on the making of a reliable hardware and software 
infrastructure which could make a remote operator able to drive 
the Morduc in comfort.
\\
Indeed, our work focuses on \textit{robot-operator interaction} and, 
hence, on how to improve such interaction. 
\\
Analyzing previous work and data produced by the supervisors' 
research group, it emerges that the stereo cameras mounted on 
top of Morduc were used to provide the remote operators a 
\textit{first-person} point of view. In literature, such a 
system is also called an \textit{egocentric} vision system.
\\
According to \cite{sugimoto}, \textit{by observing the camera image 
without an efficient human interface system, the operator 
tends to misinterpret the robot's position and direction}. This is 
due to the fact that \textit{it's difficult for an
operator not accustomed to the vehicle to estimate the
vehicle's position and direction and the distances to a
target strictly based on camera images from the first person 
viewpoint}.
\\
In order to improve the interaction between the robot and the operator 
an exocentric camera would be effective since it would provide a 
view of the robot in the operating environment and, thus, 
a better understanding of where the robot is located into the
environment and its actual direction.
\\
Unfortunately the use of an exocentric camera is not straightforward: 
for example, it could be mounted on a rear-mounted protuberance of the 
robot, but such a protuberance would terribly limit the robot activity and 
its moving abilities.
\\
To avoid such complications, \cite{sugimoto} proposes 
\textit{Time Follower's}, an approach to provide a \textit{virtual exocentric 
view} of a mobile robot.
