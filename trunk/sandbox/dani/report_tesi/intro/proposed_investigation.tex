\subsection{Proposed investigation}
\label{sec:proposed_investigation}

The project developed at the University of Hertfordshire,
UK, during period of three months, was focused to implement
a new way of teleguiding 3morduc from a remote station.
\\
The final result was a framework, called \textit{R.E.A.R.},
able to provide an external point of view (i.e. an `exocentric
vision') from which the operator
can watch and control the robot.
\\
Thanks to an external point of view, the operator can easier
perceive the environment the robot moves in. Distances between
robot and obstacles are immediately recognized, since the latter
are compared with robot's dimension.
\\
The external point of view would need another camera situated
on the rear part of the robot. We will try to `simulate' this
camera with the images provided by the frontal (and unique)
camera.
\\
Besides, the code represents a framework because it defines the basic
steps to implement an exocentric vision, regardless of how the
robot used, the communication model implemented, and other
specific detail.
\\
Before presenting deeper details, it must be underlined that
all the basic work has been accomplished together with the
student Loris Fichera, without whom \textit{R.E.A.R.} and even
this document would not be written.
\\
Next section will better introduce the reader to the problem with
appropriate discussions and examples. After describing
the author's intention, several chapter about \textit{how} the
target has been reached will follow.
\\
First effort was directed to find a proper 3morduc simulator,
to move first steps in our project: with a simulator is easier
change the environment and edit some features that would require
much more time and efforts with the real server. Test cases are
indeed faster, and produce code more reliable. Further details
can be found in chapter \ref{sec:simulator}.
\\
A new version of \textit{R.E.A.R.} has been then written in order
to use the real 3morduc robot, by the communication system exposed
in chapter \ref{sec:3morduc:communication}. Since \textit{R.E.A.R.}
was designed to be a framework, only a limited number of classes
had to be added. The authors' beginning target was fully achieved:
adding a concrete implementation of a particular robot, with a
proper communication channel, does not require built the all
the program from scratch.
\\
Finally, it is important to underline that even the main algorithm
can be changed regardless of all the other component. So, the
algorithm used with a simulator can be used when the system
communicate with real robot.
\\
All these properties increasely grow some good software properties,
like reusability, reliability and adaptability. Nevertheless,
since the code written in C++ is supported only by free libraries, it
is fully platform independent.
