\subsection{Proposed investigation}
\label{sec:proposed_investigation}

The project developed at the University of Heartfordshire,
UK, during period of three months, was focused to implement
a new way of teleguiding 3morduc from a remote station.
\\
The final result was a framework, called \textit{R.E.A.R.},
able to provide an external point of view (i.e. an `exocentric
vision') from which the operator
can watch and control the robot.
\\
Thanks to an external point of view, the operator can easier
perceive the environment the robot moves in. Distances between
robot and obstacles are immediately recognized, since the latter
are compared with robot's dimension.
\\
The external point of view would need another camera situated
on the rear part of the robot. We will try to `simulate' this
camera with the images provided by the frontal (and unique)
camera.
\\
Besides, the code represents a framework because it defines the basic
steps to implement an exocentric vision, regardless of how the
robot used, the communication model implemented, and other
specific detail.
\\
Before presenting deeper details, it must be underlined that
all the basic work has been accomplished together with the
student Loris Fichera, without whom \textit{R.E.A.R.} and even
this document would not be written.
\\
Next section will better introduce the reader to the problem with
appropriate discussions and examples. After describing
the author's intention, several chapter about \textit{how} the
target has been reached will follow.

% LocalWords:  Heartfordshire
