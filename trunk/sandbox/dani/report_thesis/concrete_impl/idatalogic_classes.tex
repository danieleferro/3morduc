\section{Classes implementing IDataLogic interface}
\label{concr:idatalogic}

In this section will be exposed classes which inherit
from \textit{IDataLogic} interface (exposed in details
in chapter \ref{rear:interfaces:idatalogic}).
\\
They differ on the way images
and robot's status data are collected: a concrete class
will use log files saved in a prefixed location on disk,
another one will ask the server to send new images and data
thought Internet network, to implement a real-time teleguide.

\subsection{DataLogicLogSimulator class}
\label{concr:idatalogic:datalogiclogsimulator}

For the reasons explained in section \ref{log:morduc_simulator},
we used a \morduc{} simulator in order to generate and record 
test data to use during off-line tests.
\\
For every simulation session, such data consist of:

\begin{itemize}
  \item a set of captured images, encoded in PNG format,
    with 632x453 pixels for each
  \item a log file reporting sampled odometry data
    for all the session
\end{itemize}

The the log file format is described in section 
\ref{log:morduc_simulator}.
Concrete \texttt{DataLogicLogSimulator} class will, hence, provide 
robot's odometry data and snapshots simply by reading 
those files.
\\
Let us have a look at its declaration:
\\
\begin{lstlisting}[caption={\texttt{DataLogicLogSimulator} declaration},
    label={code:datalogiclogsimulator:declaration}]
class DataLogicLogSimulator : public IDataLogic
{
 private:
  std::vector<image_data> _images_collection;
  int _index;
  int _simulation_session;
  
 public:
  DataLogicLogSimulator(int);
  ~DataLogicLogSimulator();
  void Command(int);
  void RetrieveData(robot_data *);
  void SelectImage(robot_data *, image_data *,
		   IImageSelector *);
};
\end{lstlisting}

Upon instantiation, \texttt{DataLogicLogSimulator} objects needs to be 
configured with a session identifier, to pass to 
the constructor. Log data must be stored within the relative path 

\begin{center}
  \texttt{../log/log\_$<$\_simulation\_session$>$}
\end{center}

with respect to the actual execution path.
Such directory must contain image and text files named and formatted
in according to what stated in section \ref{log:morduc_simulator}. 
\\
Let us now have a look at how \texttt{DataLogicLogSimulator} implements 
\texttt{IDataLogic} methods:
\\
\begin{lstlisting}[caption={\texttt{DataLogicLogSimulator::Command() method}},
    label={code:datalogiclogsimulator:command}]
void DataLogicLogSimulator::Command(int command) 
{

  // sends the command to the robot

  // in our case, just 
  // increase index to point the next line of the file
  _index++;
}
\end{lstlisting}

The \texttt{Command()} method is pretty simple in this 
implementation. Since \texttt{DataLogicLogSimulator} has been designed 
for offline testing, there's no actual command to send to the robot.
For this reason \texttt{Command()} will not take into account
the received command, the only thing done is 
incrementing the private attribute \texttt{\_index}, which 
will be used by \texttt{RetrieveData()} to decide which line 
of the log file is to read next.
\\
\begin{lstlisting}[caption={\texttt{DataLogicLogSimulator::RetrieveData()} method},
    label={code:datalogiclogsimulator:retrievedata}]
void DataLogicLogSimulator::RetrieveData(robot_data * data)
{
  // pointer to text file containing data
  FILE * position_data;
  char line[50];
  
  std::string * line_read;
  std::string line_values[4];
  std::string position_data_name;
  std::ostringstream o;
  
  int time;
  int line_number = _index;

  // grabbed image metadata
  image_data grabbed_frame_data;
  
  o << "../log/log_" 
    << _simulation_session 
    << "/data_" 
    << _simulation_session << ".txt";

  position_data_name = o.str();

  position_data = fopen(position_data_name.c_str(), "rt");

  while(fgets(line, 50, position_data) &&
	line_number > 1)
    {
      line_number--;
    }

  line_read = new std::string(line);

  std::string buf;
  std::stringstream ss(*line_read);

  // Create vector to hold our words
  std::vector<std::string> tokens;
  
  // put token in vector element
  while (ss >> buf)
    tokens.push_back(buf);

  data->x = atof ( tokens[0].c_str() );
  data->y = atof( tokens[1].c_str() );
  data->theta = TO_DEGREES(- atof ( tokens[2].c_str() ));
  data->time = atof ( tokens[3].c_str() );

  time = (int) data->time;

  // clear the stream and 
  // add a new value to it
  o.str("");
  o.clear();
  o << "../log/log_" 
    << _simulation_session 
    << "/screenshot_" 
    << _simulation_session 
    << "_" << time << ".png";

  // fill grabbed frame metadata
  grabbed_frame_data.x = data->x;
  grabbed_frame_data.y = data->y;
  grabbed_frame_data.theta = data->theta;
  grabbed_frame_data.time = data->time;

  strcpy(grabbed_frame_data.path, o.str().c_str());

  // store the collected metadata 
  // if it's not already stored
  for (std::vector<image_data>::iterator it =
	 _images_collection.begin();
       it != _images_collection.end();
       it++)
    {
      if ( (*it).time == grabbed_frame_data.time )
	{

	  return;
	}
    }

  _images_collection.push_back(grabbed_frame_data);
  return;
}
\end{lstlisting}

What the \texttt{RetrieveData()} does is just read the line - within the odometry
text log file - identified by the \texttt{\_index} private attribute of the class.
Once read, the line is parsed and the odometry data contained 
in it is used to fill the \texttt{robot\_data} passed as argument, in 
order to return the \textit{current} robot position.
\\
Be careful that last operation conceals another important duty: cast the
XY coordinates and the orientation angle (along the Y axis) returned
by the simulator environment in the proper corresponding values in OpenGL's
reference system. Lines from 45 to 47 in listing
\ref{code:datalogiclogsimulator:retrievedata} shows, in this case,
that X and Y value are not to be changed, but, on the contrary, the robot's
rotation angle has to be firstly negated (by changing its sign) and
than converted from radiants to degrees. This proves that simulator's coordinate
system differs from the one adopted in OpenGL, because not only angles
are handled with different measure scale (radiants and not degrees), but the
sense of rotation is one opposite to the other.
\\
Behind the curtain, \texttt{RetrieveData()} also creates a new 
\texttt{image\_data} structure and fills it with data just 
gathered and the relative path to the real image file (once again
thanks to the \texttt{\_index} value). Then the structure is added
to the \texttt{\_image\_collection} private attribute.
\\
This way, every time a new line of the file is read, and, hence, 
every time a new captured egocentric image is available, \texttt{DataLogicLogSimulator} 
is made able to keep track of it using its metadata, without actually 
loading the image - since it would be useless and time-consuming.
\\
\begin{lstlisting}[caption={\texttt{DataLogicLogSimulator::SelectImage()} method},
    label={code:datalogiclogsimulator:selectimage}]
void DataLogicLogSimulator::
SelectImage(robot_data * robot_status,
            image_data * bg_image_data,
            IImageSelector * selector)
{

  // since our data are already stored with vector,
  // we simply pass its reference
  selector -> ChooseImage(robot_status, 
                          bg_image_data, 
                          & _images_collection);
}
\end{lstlisting}

Finally, \texttt{SelectImage()} calls the \texttt{ChooseImage} method 
on an object of type \texttt{IImageSelector}, and passes it, as last 
parameter, a reference to its internal \texttt{\_images\_collection}, 
so that the selector can actually select the image to set as 
background.


\subsection{DataLogicLogMorduc class}
\label{concr:idatalogic:datalogiclogmorduc}

As the one exposed before, \texttt{DataLogicLogMorduc} class
fetches new images and data from static file
stored in a fixed directory, but in this case information are previously saved
not during the execution of a simulation but during
the execution of online teleguiding sessions.
\\
Logs can be created by any client which interacted with \morduc{}
server, as long as they observe the log format specified in section
\ref{log:morduc}.
\\
This class was created to develop in off-line mode the class
\texttt{DataLogicMorduc}, that by Internet connection makes possible
teleguiding the real robot (see following section,
\ref{concr:idatalogic:datalogicmorduc}).

As previously stated, every log set is made-up of:

\begin{itemize}
  \item a set of captured images, encoded in JPEG format,
    with 1280x480 or 640x480 pixels each
  \item a log file reporting sampled odometry data
    for all the session
\end{itemize}

\texttt{DataLogicLogMorduc} declaration follows:
\\
\begin{lstlisting}[caption={\texttt{DataLogicLogMorduc} declaration},
    label={code:datalogiclogmorduc:declaration}]
class DataLogicLogMorduc : public IDataLogic
{
 private:

  
  std::vector<image_data> _images_collection;
  int _index;
  int _log_session;
  int _index_max;

  FILE * odom_file;
  FILE * img_file;

  robot_data _last_robot_data;
  std::string _last_image_path;

  // standard libjpeg structures
  struct jpeg_decompress_struct _decomp_cinfo;
  struct jpeg_compress_struct   _comp_cinfo;
  struct jpeg_error_mgr _jerr;

  unsigned char * _raw_image;
 
  void GetOdometricData(robot_data*);
  std::string GetSingleImage();

  int ReadHalfJPEGFile();
  int WriteJPEGFile();

  
 public:
  DataLogicLogMorduc(int);
  ~DataLogicLogMorduc();
  void SelectImage(robot_data *, image_data *,
		   IImageSelector *);

  void RetrieveData(robot_data *);
  void Command(int);

};
\end{lstlisting}

Some private fields, such as
\texttt{\_log\_session}, \texttt{\_image\_collection}
and \texttt{\_index}, play
the same roles as the ones seen in \texttt{DataLogicLosSimulator}
class. In order, they are used to identify what log set has
to be read; to collect images retrieved by the log files (each
one represented by a \texttt{image\_data} struct);
to store with an integer how much information log has already
been consumed.
\\
Log data must be saved in the following path

\begin{center}
  \texttt{../log\_morduc/log\_$<$\_log\_session$>$}
\end{center}

always with respect to the actual execution path.
\\
New field \texttt{\_index\_max} allows to avoid some extra and
useless computation or disk access. Indeed, the integer stores
the maximum number
of images and data client can read form the log set, so
\texttt{DataLogicLogMorduc}  can
return the last robot data and image previously loaded, respectively,
in \texttt{\_last\_robot\_data} and \texttt{\_last\_image\_path},
when \texttt{\_index}
reaches \texttt{\_index\_max}'s value.
\\
Some private fields and methods are specialized to work with JPEG
images and are built on \textit{libjpeg} (we refer to \cite{library:libjpeg}
for official reference) library functions.
Since \texttt{DataLogiLogMorduc} can deal with 1280x480
or 640x480 pixel images (the former contains both right and left
camera images, whereas the latter only the left or right one),
some supporting methods and variables are needed.
\\
When new image data are requested, \texttt{DataLogiLogMorduc}
calls method \texttt{GetSingleImage()}, which, based on the
\texttt{\_index}'s value, reads the next image from the log set.
\\
If image's dimension is 1280x480, it invokes in turn the private method
\texttt{ReadHalfJPEGFile()} in order to load only half 
picture (i.e. 640x480 pixels) in memory space pointed by variable
\texttt{\_raw\_image}. Afterwards, by calling \texttt{WriteJPEGFile()},
data are read from memory and saved to static file overwriting the
original image one the disk.
\\
In this way next time the same log session is selected \texttt{DataLogicLogMorduc}
will find a 640x480 pixel JPEG image, hence no further computation
or write operation on disk has to be performed.
\\
Obviously, when the image indicated by \texttt{\_index} value is
not found, or with a dimension different from 1280x480 or
640x480 pixels, an error is reported and the application terminate.
\\
We can now proceed to see how \texttt{DataLogicLogMorduc} implements
\texttt{IDataLogic} interface methods.
\\
The \texttt{Command} method, as seen for the \texttt{DataLogicLogSimulator}
class, is pretty simple: it has to increment \texttt{\_index},
which indicate the next line in \texttt{odometric.txt}
text file to read and the next image to open. It is incremented until
its value is equal to \texttt{\_index\_max}.
\\
\begin{lstlisting}[caption={\texttt{DataLogicLogMorduc::Command() method}},
    label={code:datalogiclogmorduc:command}]
void DataLogicLogMorduc::Command(int command) {

  if (_index <= _index_max)
    _index++;
}
\end{lstlisting}

The \texttt{RetrieveData()} method is displayed in listing
\ref{code:datalogiclogmorduc:retrievedata}:
\\
\begin{lstlisting}[caption={\texttt{DataLogicLogMorduc::RetrieveData()} method},
    label={code:datalogiclogmorduc:retrievedata}]
void DataLogicLogMorduc::RetrieveData(robot_data * data) {

  image_data grabbed_frame_data;
  std::string image_path;

  /* STEP 1: retrieve data from file (server) */
  GetOdometricData(data);

  /* STEP 2: get single image path */
  image_path = GetSingleImage();
   
  // fill grabbed frame metadata
  grabbed_frame_data.x = data->x;
  grabbed_frame_data.y = data->y;
  grabbed_frame_data.theta = data->theta;
  grabbed_frame_data.time = data->time;
  strcpy(grabbed_frame_data.path, image_path.c_str());

  /* STEP 3: store the collected metadata
     if it's not already stored */
  for (std::vector<image_data>::iterator it =
	 _images_collection.begin();
       it != _images_collection.end();
       it++)
    {
      if ( (*it).time == grabbed_frame_data.time )
	{
	  return;
	}
    }

  _images_collection.push_back(grabbed_frame_data);
  
  return;
}
\end{lstlisting}

As the reader noted, \texttt{RetrieveData()} parts its duties in
three steps. First step is accomplished by calling the private method
\texttt{GetOdometricData()}, which will open the odometry file, read
the line indicated by \texttt{\_index} and extract the information
contained in the latter to fill the \texttt{robot\_data} structure
pointed by its unique argument. The format of each text line written in
\texttt{odometric.txt} can be found in section \ref{log:morduc}.
\\
\texttt{GetOdometricData()} is also responsible for mapping XY
coordinates and rotation angle, retrieved by the log
odometry file written with \morduc{} server data, in the matching values
to provide to the OpenGL's reference system. The XY coordinates returned
by \morduc{} are expressed in meters and have to be increased hundred
times in order to produce a meaningful robot's movement within the
synthesized OpenGL space. A single unit of measure in the latter stands
in fact for one centimetre.
\\
Furthermore, the coordinate conversion process (performed by  
\texttt{GetOdometricData()} and hence not shown in listing 
\ref{code:datalogiclogmorduc:retrievedata}) inverts the Y axis sense
by changing the sign of every Y value read from log; at last, the
rotation angle theta is converted from radiant to degrees, because
\morduc{} and OpenGL use different notations.
\\
To sum up, the coordinate system used by \morduc{} server presents,
in reference to the one adopted by OpenGL, 1. a smaller unit of
measure, 2. the Y axis increasing with opposite sense, 3. rotation
angle expressed in radiants and not in degrees.
\\
Note that matching the source data coordinate system in the OpenGL one
depends heavily on the  kind of source we are dealing with. In previous
section, where data were collected from the simulator log files, conversion
was
performed in a totally different way because simulator implements
its own coordinate system.
\\
Back to \texttt{RetrieveData()} method, in second step a call to
\texttt{GetSingleImage()} is performed,
returning a struct pointing to the new image.
At last, the previous \texttt{image\_data} struct is insert to the set
of available images, by adding it to the \texttt{\_image\_collection}
vector.
\\
\texttt{SelectImage}'s code is completely equal to the one
presented in the previous section. It refers whole the computation to the
\texttt{ChooseImage()} method implemented by the instance pointed by
its last argument.
\\
\begin{lstlisting}[caption={\texttt{DataLogicLogMorduc::SelectImage()} method},
    label={code:datalogiclogmorduc:selectimage}]
void DataLogicLogMorduc::
SelectImage(robot_data * robot_status,
            image_data * bg_image_data,
            IImageSelector * calculator) {

  calculator->ChooseImage(robot_status, bg_image_data,
                          &_images_collection);
}

\end{lstlisting}

\texttt{ChooseImage()} will select the proper image to render as texture
by knowing the actual robot position (fist parameter) and the set of available
image shot by the robot during its path. The chosen image is returned by filled
the structure pointed by its second parameter.
\\
Section \ref{concr:iimageselector} will cover in details algorithms and
features implemented by different kind of \texttt{ChooseImage()} methods.

\subsection{DataLogicMorduc class}
\label{concr:idatalogic:datalogicmorduc}

This class allows \framework{} to teleguide the real \morduc{}
robot, situated at DIEES laboratory, in Catania (Italy).
\\
Several \texttt{DataLogicMorduc}'s features originate 
from \texttt{DataLogicLogMorduc} class, exposed earlier in
\ref{concr:idatalogic:datalogiclogmorduc}. Main differences
rely on the way new data are retrieved, since they are not
statically saved in log file but dynamically created and sent
from the server to the client through the Internet network.
\\
\texttt{DataLogicMorduc} can be thought of as the final
stage of a development project whose primary aim was
to teleguide a real mobile robot by remote control.
\\
The
robot concerned is the mobile-platform \morduc{}, with which
we can
establish a communication channel through simple
HTTP requests and responses. Details about accepted
commands and returned replies can be found in 
section \ref{intro:3morduc:communication}; here we only
remind briefly that \morduc{} can be controlled
by requesting different resources (i.e. URLs).
\\
A set of specific URLs allows to drive the robot in remote
environment, changing its position forward or backward and
steering in left or right direction. Another set
of resources are instead provided to simply retrieve robot's
data without altering its status.
\\
\texttt{DataLogicMorduc} class bases its functionalities
on two fundamental libraries, named \textit{libjpeg} and
\textit{curlpp}. The former, already mentioned previously,
offers the means for handling JPEG images, which is the codec
server compresses the images with before sending them to
client.
The latter is a C++ wrapper for the widely used
\textit{libcurl} (\cite{library:libcurl}). It exposes some
classes and data structure
that allow sending HTTP request and processing the relative
response within few code lines.
\\
Official references about libraries cited above can be
found in \cite{library:libjpeg} (\textit{libjpeg}) and
\cite{library:curlpp} (\textit{curlpp}).
\\
\texttt{DataLogicMorduc}'s declaration is reported in
listing \ref{code:datalogicmorduc:declaration}.
\\
\begin{lstlisting}[caption={\texttt{DataLogicLogMorduc} declaration},
    label={code:datalogicmorduc:declaration}]
class DataLogicMorduc : public IDataLogic
{
  

 private:

  std::vector<image_data> _images_collection;
  int _index;
  
  FILE * odom_file;
  FILE * img_file;

  std::string _path_server;
  std::string _path_data;

  HTTPFunctor _http_functor;
  
 public:
  DataLogicMorduc(const char*, const char*);
  ~DataLogicMorduc();
  void SelectImage(robot_data *, image_data *,
		   IImageSelector *);

  void RetrieveData(robot_data *);

  void Command(int);

};
\end{lstlisting}

As requested, it implements \texttt{IDataLogic} interface and
hence presents its three public methods along with class constructor
and destructor.
\\
To create a new class instance, client has to specify
two string parameters. Their values indicate, in order, the IP
address or URL where the server is located and the path of an
existing folder. \texttt{\_path\_server} and \texttt{\_path\_data}
private fields help to store them, since their value are needed
several times during object life.
\\
\texttt{\_path\_data} is used to save data provided by the server during the
teleguiding session. \framework{} needs to store images obtained
in online mode in order to simulate the \textit{exocentric} point
of view: an image shot earlier by the egocentric camera must be
selected by the core algorithm every time a robot change its position.
The chosen image is then employed as texture by OpenGL, with the aim
to recreate the real world within virtual space.
\\
Images are saved in the specified path, each one according to the
following notation, thanks to the \texttt{img\_file} pointer
shown in previous listing.

\begin{center}
  \texttt{$<$user\_path$>$/img$<$index\_value$>$.jpg}
\end{center}

Obviously, integer \texttt{\_index} starts counting from one and is
incremented every time new image and robot data are received by the
client. By means of \textit{libjpeg} library, image obtained is
firstly loaded in memory (i.e. in dynamic memory), and then only half
its content its written to hard disk: \framework{} exploits only
left or right camera image, but server returns (through HTTP body
response) both images through a single 1280x480 pixels JPEG image,
so \texttt{DataLogicMorduc} saves directly 640x480 picture in order
to avoid future further computation.
\\
Besides, each image is represented by a \texttt{image\_data}
structure collected in vector \texttt{\_images\_collection}. This
collection allows to quickly access information regarding the
available images, since every element can be thought of as a pointer
to the real coupled image (we remind that every struct contains, among
other information, the absolute path referring to the image).
\\
Even thought \framework{} would not need to save other data aside
from pictures frame, \texttt{DataLogicMorduc} creates, always on
the specified directory, a text file containing odometry data retrieved
along with each image. Its absolute path will be:

\begin{center}
  \texttt{$<$user\_path$>$/odometric.txt}
\end{center}

File pointer \texttt{odom\_file} serves for the purpose. Storing
odometric information proves to be extremely useful because,
together with camera images, they form new log data client can
based its future computation on by using \texttt{DataLogiLogMorduc}
class. The advantages of log data (formatted in a standard way)
has been widely exposed through
this document, for details see section \ref{log}.
\\
Last, but not least, \texttt{\_http\_functor} field
collects data coming from Internet network, that is the response sent from
server. It performs the role of a `functor', deeper explained in
following subsection (\ref{concr:idatalogic:datalogicmorduc:httpfunctor}).
\\
As usual, we can now proceed to examine in detail how \texttt{IDataLogic}
virtual method are implement. To begin with, \texttt{Command()} code is
displayed.
\\
\begin{lstlisting}[caption={\texttt{DataLogicMorduc::Command() method}},
    label={code:datalogicmorduc:command}]
void DataLogicMorduc::Command(int command) {

  // send the command to the robot
  std::string command_URL;
  std::ostringstream os;

  switch (command) {

  case FORWARD:
    command_URL = _path_server + "stereo.fow1.jpg";
    break;

  case BACKWARD:
    command_URL = _path_server + "stereo.bak1.jpg";
    break;

  case RIGHT:
    command_URL = _path_server + "stereo.rgt.jpg";
    break;

  case LEFT:
    command_URL = _path_server + "stereo.lft.jpg";
    break;

  default:
    std::cout << "Unknown command: " << command
	      << std::endl << "Program will terminate."
	      << std::endl;

    exit(1);
  }
 
  try {

    curlpp::Cleanup cleaner;
    curlpp::Easy request;

    // header options' list
    std::list<std::string> headers;
    headers.push_back("User-Agent: REAR client"); 

    // Setting request options
    request.setOpt(new curlpp::options::Url(command_URL));
    request.setOpt(new curlpp::options::Verbose(false));
    request.setOpt(new curlpp::options::HttpHeader(headers));
    // redirect response to os
    request.setOpt(new curlpp::options::WriteStream(&os));    
    
    // send request
    request.perform();    
  }
       
  // increase index to point the next line of the file
  _index++;

}
\end{lstlisting}

Unlike the other class implementing \texttt{IDataLogic} interface
exposed so far, \texttt{DataLogicMorduc}'s \texttt{Command()} method takes
in account its integer parameter because its final aim is to
\textit{really} teleguide the robot: next robot's status it is
not been determinated and written in a log file, but it depends
on which command the teleoperator asks to execute.
\\
For one of each available command (go forward, go backward, turn right
and turn left) a specific resource must be requested to the connected
\morduc{} server. URL is built by merging server's address (stored
in \texttt{\_path\_server}) with a string indicating the requested
command (see chapter \ref{intro:3morduc:communication} for a complete
list of all possible values).
\\
After constructing URL, \texttt{Command()} prepares an HTTP
request by means of \textit{curlpp} objects (lines 35 to 50 in listing
\ref{code:datalogicmorduc:command}). Therefore, the request is sent
simply by invoking the \texttt{perform()} method of the
\texttt{curlpp::Easy} object. This few lines of code evince
\textit{curlpp}'s potential to largely reduce programmer's effort in
dealing with HTTP protocol (as well as many other Internet protocol,
\cite{library:curlpp}).
\\
At last, \texttt{\_index} value is incremented to uniquely identify
next saved image.
\\
\texttt{RetrieveData()} subdived its operations in three steps. Its
code is shown in following listing, but in order to
focus reader's attention on chief points, some control code (e.g. checking
if opening a file subroutine returns without errors) has been deleted.
\\
\begin{lstlisting}[caption={\texttt{DataLogicMorduc::RetrieveData()} method},
    label={code:datalogicmorduc:retrievedata}]
void DataLogicMorduc::RetrieveData(robot_data * data) {

  std::ostringstream o;
  // img path
  o << _path_data << "img" << _index << ".jpg";

  std::string command_URL = _path_server + "stereo.jpg";
  std::string image_path = o.str();
  std::string line_read;

  image_data grabbed_frame_data;
  
  // STEP 1: REQUEST DATA FROM SERVER

  try {

    curlpp::Cleanup cleaner;
    curlpp::Easy request;

    _http_functor.ResetFunctor();
    
    // Set the writer callback to enable cURL 
    // to write result in a memory area
    curlpp::types::WriteFunctionFunctor
                   functor_body(&_http_functor, 
		                &HTTPFunctor::
                                WriteHTTPBodyCallback);
      
    curlpp::types::WriteFunctionFunctor
                   functor_header(&_http_functor, 
				  &HTTPFunctor::
                                  WriteHTTPHeaderCallback);
    
    // header options' list
    std::list<std::string> headers;
    headers.push_back("User-Agent: REAR client"); 
      
    // Setting request options
    request.setOpt(new curlpp::options::
                     Url(command_URL));
    request.setOpt(new curlpp::options::
                     Verbose(false));
    request.setOpt(new curlpp::options::
                     WriteFunction(functor_body));
    request.setOpt(new curlpp::options::
                     HeaderFunction(functor_header));
    request.setOpt(new curlpp::options::
                     HttpHeader(headers));
        
    // send request
    request.perform();
      
    line_read = _http_functor.GetOdometryString();
    _http_functor.CreateImage(image_path.c_str());
  }
  
  // STEP 2: WRITE ODOMETRIC DATA TO FILE
  // AND FILL ROBOT DATA
  
  // open new file per odometry data
  std::string odom_file_path = _path_data  + "odometric.txt";
  odom_file = fopen(odom_file_path.c_str(), "at");  

  fputs(line_read.c_str(), odom_file);
  fclose(odom_file);

  FillOdometricData(line_read, data);

  // STEP 3: insert image in collection
   
  // fill grabbed frame metadata
  grabbed_frame_data.x = data->x;
  grabbed_frame_data.y = data->y;
  grabbed_frame_data.theta = data->theta;
  grabbed_frame_data.time = data->time;

  strcpy(grabbed_frame_data.path, image_path.c_str());

  // store the collected metadata if it's not already stored
  for (std::vector<image_data>::iterator it =
	 _images_collection.begin();
       it != _images_collection.end();
       it++)
    {
      if ( (*it).time == grabbed_frame_data.time )
	{
	  return;
	}
    }

  _images_collection.push_back(grabbed_frame_data);
  
  return;
}
\end{lstlisting}

First of all, the URL of the resource requested is built (line 7):
in these case we want to receive only image and odometry data
(see \ref{intro:3morduc:communication} for further details about
\morduc{}'s served URLs).
\\
Next, first step set some options in a \texttt{curlpp::Easy} object
before sending the request invoking its \texttt{perform()} method. Among
these, the callback functions are indicated for managing HTTP response
header and body. Both callbacks are implemented by the functor
\texttt{HTTPFunctor} class, exposed later. When the \texttt{perform()}
method is called, a synchronous call is done and the execution proceeds
only when all response data has been received.
\\
Since HTTP header and body will be received in an asynchronous mode
(data through network are parted in package), callbacks will be
called every time a new piece of information is received.
\texttt{functor\_body} (declared from line 24) manages data belonging
to HTTP body, whereas \texttt{functor\_header} (declared from line 29)
manages data belonging to HTTP header.
\\
Functor's task is to proper store received data as long as they arrive.
When the execution resumes from the \texttt{perform()} method (line 52),
\texttt{RetrieveData()} can complete its first step by asking
to \texttt{\_http\_functor} to return the odometry string received from server
(originally contained in HTTP header) and to save image contained in
HTTP body into JPEG file (better, half the image, as previously stated).
\\
In second step odometry text line is written in \texttt{odometric.txt},
and then the same string is passed to the private \texttt{FillOdometricData()}
method in order to extract its information and fill the pointed
\texttt{robot\_data} structure. Only one note has to be done: as seen
in previous classes implementing \texttt{IDataLogic} interface, a
proper mapping between source data system reference and OpenGL coordinate
system has to be performed.
\\
In this case, the operation are the same already exposed in previous section,
for \texttt{DataLogicLogMorduc}. In synthesis, we have to 1. multiple X and Y
value by 100, 2. change the sign of  Y values, 3. change angle of rotation
measure from radiants to degrees. For further details we refer to section
\ref{concr:idatalogic:datalogiclogmorduc}.
\\
Step number three consists of the common piece of code met several
times in this document. A new \texttt{image\_data} struct, representing
the received and locally stored image, is created and inserted in
\texttt{\_image\_collection} set.
\\
Last public method, \texttt{SelectImage()} refers the chosen of the image
to put as texture completely to the passed instance of
\texttt{IImageSelector}, which will fill the pointed \texttt{image\_data}
with a element contained in \texttt{\_image\_collection}.
\\
\begin{lstlisting}[caption={\texttt{DataLogicMorduc::SelectImage()} method},
    label={code:datalogicmorduc:selectimage}]
void DataLogicMorduc::
SelectImage(robot_data * robot_status,
            image_data * bg_image_data,
            IImageSelector * calculator) {

  calculator->ChooseImage(robot_status, bg_image_data,
                          &_images_collection);
}

\end{lstlisting}

\subsubsection{HTTPFunctor class}
\label{concr:idatalogic:datalogicmorduc:httpfunctor}

Before looking over \texttt{HTTPFunctor} class, a brief preamble
about functors in C++ is supplied.
\\
Functors can be thought of as functions with a state. In C++ they
are realized
as a class with one or more private members to store the state
and with an overloaded operator \textit{()}, in order to execute the
functor as a simple function.
\\
Besides, functors can implement callbacks.
When designing application, we often know all of the classes
with which a specific component will interact and thus we explicit its
interface in terms of those classes' type. When designing general purpose
or library components, however, it is often necessary or desirable
to put in hooks for calling unknown objects. What is required is
a way for one component to call another without having been written
in terms of, or with knowledge of, the other component's type. Such a
`type-blind' call mechanism is often referred to as a \textit{callback}.
\\
A callback, then, is a way for a component designer to offer a generic
connection point which developers can use to establish communication
with application objects. At some subsequent point, the component `calls
back' the application object. The communication takes the form of a
function call, since this is the way objects interact in C++.
\\
There are always three elements in any callback mechanism: the caller, the
callback function, and the callee.
\\
The caller is usually an instance of some class, for instance a library
component, that requires the callback; it can, or must, call some
third party code to perform its work, and uses the callback mechanism
to do so. As far as the designer of the caller is concerned, the callback
is just a way to invoke a process.
\\
The caller determines the signature of the callback function, i.e. its
arguments and return types. This makes sense, because it is the caller
that has the work to do, or the information to convey.
\\
The callee is usually a member function of an object of some class, but it
can also be a stand-alone function or static member function, that the
application designer wishes to be called by the caller component.
The function to be called must be compatible with the
signature of the callback function specified by the caller.
\\
A concrete callback mechanism example is given with \texttt{HTTPFunctor}
class. The caller, in this case, is performed by an instance of
\texttt{curlpp::Easy} class, which needs a hook point to incrementally
manage the HTTP response data received by server. Since we want to
process in two different ways bytes of information composing HTTP body
and byte composing HTTP header, two different callbacks are needed.
\\
\texttt{HTTPFunctor}, in addition to provide callbacks, owns some
private field in order to proper store the received information. Its
declaration is shown in following listing:
\\
\begin{lstlisting}[caption={\texttt{HTTPFunctor} declaration},
    label={code:datalogiclogsimulator:httpfunctor:declaration}]
class HTTPFunctor
{
 private:

  // to store jpeg image from HTTP server
  char* _m_pBuffer;

  // store header
  std::string _header;

  // store decompressed jpeg image
  unsigned char * _raw_image;

  void* Realloc(void*, size_t);
  int ReadHalfJPEGBuffer();
  int WriteJPEGFile(const char*);

 public:
  HTTPFunctor();
  ~HTTPFunctor();

  std::string GetOdometryString();
  void ResetFunctor();
  void CreateImage(const char*);

  // callback
  size_t WriteHTTPBodyCallback(char*, size_t, size_t);
  size_t WriteHTTPHeaderCallback(char*, size_t, size_t);
};
\end{lstlisting}

Callback function, declared with public access, are named
\texttt{WriteHTTPBodyCallback} and \texttt{WriteHTTPHeaderCallback}.
The
careful reader has already noticed that these name have been
encountered in previous section
(\ref{concr:idatalogic:datalogicmorduc}), where we specified
caller's hook points.
\\
As previously stated, type and number of parameters, as well
as the return value, is defined by the caller itself, which
determine also their meaning. The requested signature, in this
case, is equal for both callbacks:

\begin{center}
  \texttt{size\_t Callback(char* ptr, size\_t size, size\_t nmemb)}
\end{center}

Char pointer \texttt{ptr} indicates where new data are temporarily
stored. Bytes belonging to HTTP header or body do not arrive as a
unique block, but in separated, although ordered, chunks of bytes:
everytime one or more chunks arrive, the proper callback is invoked.
\\
Last two input parameters tell, respectively, the size of each
block of byte, memorized from the char pointer \texttt{ptr} address,
and their quantity. Since every block has the same dimension,
total amount of data to process can be computed by multiplying
\texttt{size} with \texttt{nmemb}. The last obtained value must
always be returned by the callback, regardless of the performed
computation.
\\
We can now have a look on how the previous callbacks work. The one
processing HTTP header data (listing
\ref{code:datalogiclogsimulator:httpfunctor:writehttpheadercallback})
deals with new coming bytes representing part of a string text, because
header is transmitted as simple text. All it has
to do is building a new string which will collect  characters
pointed in memory by \texttt{ptr}; at every callback invocation new
characters are inserted in append to the ones previously received.
\\
\begin{lstlisting}[caption={\texttt{HTTPFunctor::WriteHTTPHeaderCallback} method},
    label={code:datalogiclogsimulator:httpfunctor:writehttpheadercallback}]
size_t HTTPFunctor::
WriteHTTPHeaderCallback(char* ptr, size_t size, size_t nmemb) {

  // Calculate the real size of the incoming buffer
  size_t realsize = size * nmemb;
  _header = _header.append(ptr);
 
  return realsize;
}
\end{lstlisting}

The other callback, instead, has to deal with bytes composing
a JPEG image.
\\
\begin{lstlisting}[caption={\texttt{HTTPFunctor::WriteHTTPBodyCallback} method},
    label={code:datalogiclogsimulator:httpfunctor:writehttpbodycallback}]
size_t HTTPFunctor::
WriteHTTPBodyCallback(char* ptr, size_t size, size_t nmemb) {

  // Calculate the real size of the incoming buffer
  size_t realsize = size * nmemb;
  
  // (Re)Allocate memory for the buffer
  _m_pBuffer = (char*) Realloc(_m_pBuffer, _m_Size + realsize);
  
  // Test if Buffer is initialized correctly & copy memory
  if (_m_pBuffer == NULL) {
    realsize = 0;
  }
  
  memcpy(&(_m_pBuffer[_m_Size]), ptr, realsize);
  _m_Size += realsize;
   
  // return the real size of the buffer
  return realsize;
}
\end{lstlisting}

In these case received bytes can not be handled as simple characters,
but have to be stored in a consecutive part of memory as new chunks
of information are received. For this purpose, data are copied in location
pointed by char pointer \texttt{\_m\_pBuffer}.
\\
Every time new data has to be added to the previously stored, we need
to find another part of memory large enough to accept the total amount of
bytes and assign the new starting address to \texttt{\_m\_pBuffer}. The
\texttt{Realloc} (based on \texttt{malloc} and \texttt{realloc} subroutines)
accomplishes to this task.
\\
After guaranteeing enough space in memory to accept old and incoming data,
new bytes
are added consecutively to previous one by calling the \texttt{memcpy}
subroutine. The \texttt{\_m\_Size} field indicates how much data
has already been store between a callback invocation and another.
\\
When data are all correctly received, execution control returns to the
caller, i.e. - in our case - to \texttt{curlpp::Easy} instance, which
in turn terminates its execution and returns control to the
\texttt{DataLogicMorduc} instance. By means
of the last two public methods exposed by \texttt{HTTPFunctor},
\texttt{DataLogicMorduc}
can now obtain collected information contained in HTTP response header and
body. The two methods are:

\begin{itemize}

  \item \texttt{std::string GetOdometryString()} \\
    returns only the string indicating robot's odometry data,
    by isolating it from all the header text saved in private
    variable \texttt{\_header};

  \item \texttt{void CreateImage(const char*)} \\
    allows to write on disk half the JPEG image received
    by the server (a 640x480 pixels from a 1280x480 pixels image), with
    filename indicated by its unique parameter. \\
    Its duty is performed by calling private methods
    \texttt{ReadHalfJPEGBuffer()} and \texttt{WriteJPEGFile(const char*)},
    in turn based on \textit{libjpeg} library. Similar function has
    already be exposed in section \ref{concr:idatalogic:datalogiclogmorduc},
    to present \texttt{DataLogicLogMorduc} class.

\end{itemize}

In conclusion, it is worth pointing out that if a single instance of
\texttt{HTTPFunctor} is used to compute two o more HTTP response,
its state has to be cleared before letting it received new data: on the contrary,
both HTTP header and body data will be added to ones earlier stored.
\\
Public method \texttt{ResetFunctor()} serves the scope, by 1. freeing all
data bytes maintained in memory and pointed by its char pointers; 2. clearing the
\texttt{\_header} string 3. reset \texttt{\_m\_Size} counter.
