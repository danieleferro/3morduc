\setcounter{figure}{0}
\setcounter{table}{0}
\setcounter{lstlisting}{0}

\chapter{Log files and their format}
\label{log}
\minitoc

A basic exocentric vision system for \morduc{} would need, 
at least, some images captured from the camera mounted 
on top of it and, for each image, the actual position of 
the robot at the time it was captured.
\\
While images could be used as a texture on which draw 
a 3d representation of the robot, information about 
position are essential in order to draw the robot 
consistently, i.e. in the position the observer would 
see it if he was seeing the robot by means of an actual 
external camera.
\\
Images and data can be read from different kind of source 
information. If we want to interact in `on-line' mode with
the robot, they are sent through the Internet network;
otherwise, images and robot's odometric data can be
obtained by previous log session stored into a predefined
path in our disk.
\\
It is now only briefly reported that
\framework{} can interact with different type of source
information, by means of the \texttt{IDataLogic} interface.
Further explanations will be supplied later on this in this
document (chapter \ref{rear}).
\\
In this section two different kind of log information are
exposed. The first is obtained by guiding the robot through
a simulator program, able to recreate the demanded environment
where a \morduc{} replica can move in. All images (as seen by
the robot virtual camera) and data are collected and stored
during the simulation running, so \framework{} can retrieve
its information from static files.
\\
The second category of log information \framework{} can base
its computation on are derived from previous ran 
real \morduc{} teleguiding
session.
\\
Making \framework{} able to work with information contained in
static file has proved to be highly important in development.
We do not have \morduc{} always powered on, connected to the
network and free to move in any direction, whenever
we want to test a specific feature of our client program. Besides,
the test condition we want to check our feature on could be not easy
or immediate to recreate in real word, for whatever physical
impediment or constriction. Moreover, it could happen that a specific
robot
path must be repeated more times, always the same, to verify
different proposed solutions.
\\
The core algorithm encapsulated in \framework{} can be test
on real online session as well as with offline data, that 
makes all the development phase easer.

\clearpage
\section{Morduc Simulator Log}
\label{log:morduc_simulator}

An exocentric vision systems performs at its best 
when there's a large static space where the robot can move in. Since 
the real robot can not be easily teleguided in wide 
environment because of its size and the lack of a large 
room in Catania, where the robot is situated, a simulator 
has been used to reproduce the best set of data. The 
simulator can be thought of as a server, which receives 
requests and returns responses: the first are the commands 
sent by the user to move the robot, the latter the 
egocentric images and the robot position data.
\\
Furthermore, with a simulator is extremely easy to 
change the environment where the robot is teleguided, 
so we can test the exocentric vision with an infinite 
number of environments without physically moving the 
robot in different places. In this way software 
development of exocentric vision can be faster, because 
it is simple and immediate to establish several test cases.
\\
The first simulator adopted was \textit{Rosen} (\cite{rosen}). Written in 
\textit{Erlang} \cite{erlang}, \textit{Rosen} has been developed at
the \textit{University of Catania} in order to simulate the behaviour of
autonomous mobile robots (AMRs).
\\
\textit{Rosen} has been used as test bench for robots taking part in 
the \textit{Eurobot} competition \cite{eurobot}, and hence, 
for robots with completely different features from the ones of \morduc{}. 
\\
As soon as we realized \textit{Rosen} does not meet our needs, 
we started looking for a more suitable simulator.
\\
In 2006, at the \textit{Aalborg University}, Filippo 
Privitera wrote a simulator specifically intended for 
the \morduc{} platform \cite{privitera}. 
\\
Such a simulator reproduces the \morduc{} itself (actually a 
3d model of it) situated in a customizable room: the position 
of walls can be specified by the user, by giving 
the simulator a black and white bitmap image with the room 
planimetry, to build the whole environment from. 
\\
Besides, Privitera's simulator allows to enable the 
stereoscopic vision, with anaglyph or polarized method (both 
types are applied on the egocentric camera). Other informations 
like the number of collisions or the robot distance from the 
nearest obstacle are provided by simulator.
\\
The simulator was written using the Microsoft Foundation Class
(MFC) framework and has been used for testing user ability in
tele operating driving a robot, in comparison with the actual 
robot to quantify the differences between the two facilities. 
\\
With a simulator specifically built for the \morduc{} robot, it 
was not difficult to edit the source code to obtain what we 
needed. First of all, we needed to record data about egocentric 
vision and robot status, because they are the input value 
of the exocentric vision simulator. In order to achieve this, 
we edited the source code to allows the user to store data: by 
pressing the 'P' key keyboard the actual information (e.g. the 
actual camera image and robot status) are recorded in log files.
\\
Every session in Privitera's simulator has its own identifier 
(a integer number). When the 'P' key is pressed the simulator 
write a new line in the text file named

\begin{center}
  \textit{data\_$\langle$number of session$\rangle$.txt}
\end{center}

creating the file if it does not 
exist. Each line contains four float number values, with the 
following meaning:

\begin{enumerate}
\item x coordinate
\item y coordinate
\item theta value (in radiants)
\item timestamp
\end{enumerate}

where the timestamp refers to the beginning of the simulation.
\\
Beside the text file there are several PNG images, each for every 
line written in \textit{data\_$\langle$number of 
session$\rangle$.txt}.
These files are named

\begin{center}
  \textit{screenshot\_$\langle$number of 
    session$\rangle$\_$\langle$timestamp$\rangle$.png}
\end{center}

where the 
number of session indicates for every screenshot 
- i.e. the egocentric vision - the proper text file, and the 
timestamp the line with the associated status of the robot.
\\
The software which implement the exocentric vision control 
will look for text and image files related to a specific 
session, in order to read the necessary input and draw the 
robot correctly. It must be able to choose the right image 
to use as background, among those previously read; to draw 
the robot over the background in the right position and 
orientation, depending on its route; to prevent or signal 
collisions to the user, and so on.


\clearpage
\section{Morduc Log}
\label{log:morduc}

Different working groups have used \morduc{} to implement their
specific client. A large amount of log data
information has so been collected from real sessions, in
order to replicate offline all the state hold by the robot
during previous teleguiding runs.
\\
Because \morduc{} server does not implement any functionality
to save log data on its disk, clients which wished to store
images and data collected during online runs have developed their
proper piece of code to create log files locally.
Even though every client might have created log files with a proper and
specific format, every one followed a specific but not official
standard, described in this section.
\\
Using a common way to store data allows older and future client to
test their functionality on a wide set of log, without
worrying about which client collected the information:
changing the client does not require to change data log,
thanks to the standard format.
\\
Another advantages of this approach 
its that the server is not overloaded
with log data, because they are stored locally on client.
\\
For our tests we can rely on two text files and several images.
Usually client add new data to the log file set 
when robot changes its status,
i.e. its position, after sending a teleguide command.
\\
Cameras' image are saved in 
a single 1280x480 pixel JPEG image, composed by two
640x480 pixel images, one from the right and one from the left camera.
A counter variable,
starting from number one and incremented every time a new image
has to be written, is used in order to assign each picture a different
name according to the following notation:

\begin{center}
  \texttt{img\_$\langle$counter\_value$\rangle$.jpg}
\end{center}

Every created image is coupled with two different type of
information, collected in the shooting instant and providing,
respectively, robot's odometric and scanner laser data.
\\
The former is saved in text file named

\begin{center}
  \texttt{odometric.txt}
\end{center}

where the line i-th contains the odometric data tied to the i-th
images. Every line is copied from the HTTP response header sent
by the \morduc{} server (see section \ref{intro:3morduc:communication}),
so the format will be

\begin{center}
  \texttt{Morduc/$\langle$time$\rangle$$\backslash$$\langle$x$\rangle$$\backslash$$\langle$y$\rangle$$\backslash$$\langle$theta$\rangle$$\backslash$$\langle$collision\_number$\rangle$$\backslash$$\langle$min\_distance$\rangle$}
\end{center}

obviously, the text file will contain as many lines as many
saved images.
\\
Laser data are saved, instead, in file named

\begin{center}
  \texttt{laser.txt}
\end{center}

also in this case line i-th is coupled with the i-th image, and
every line is rewritten from text data read in HTTP header
response:

\begin{center}
  \texttt{Laser/$\langle$value\_1$\rangle$/$\langle$value\_2$\rangle$/../$\langle$value\_181$\rangle$}
\end{center}

Since laser scans 180$\textdegree$ with 1$\textdegree$ step,
181 values are saved.
\\
Even though \framework{} do not takes
advantages of laser information, these could be used by future
version in order to improve operator skill in guiding the robot
(e.g. by indicating distance from nearest object with augmented
reality, as implemented in \cite{morduc:macalusodetommaso}).

