\section{Perspective in OpenGL}
\label{opengl:perspective}

Perspective in OpenGL can be something very trick to work with. 
It all starts from the definition of \textit{frustum}, as the portion
of space seen from the point of view (see \cite{wiki:frustum} 
for further information). To define the frustum we use the function
named \texttt{gluPerspective()} \cite{opengl:gluPerspective}, which takes 
four parameters.
\\
The first parameter indicates the \textit{FOV} - i.e. Field Of View -
along the Y axis, in degrees: the chosen value is 60. The
second one expresses ratio between actual width and height, 
in our case 624/442. The last two parameters specify the distance 
from the viewer to the near clipping plane and to the far one, 
according to the frustum definition.
\\
By using two close values for the far and near clipping planes 
distance the application will not be able to display 3d
objects, unless the point of view is relatively close to them. 
If we want to be able to see 3d objects as they are moving away
from the point of view, we need to increase the difference between 
the last two parameters of the \texttt{gluPerspective()} function. In
our case the chosen values are 0.001 and 100000 (its ratio is 
equal to 10exp8).
\\
The \texttt{gluPerspective()} function affects the \textbf{projection} 
matrix, previously selected by changing matrix mode. Finally,
in order to set our point of view, we must change matrix again, 
this time to work with the \textbf{modelview} matrix.
\\
The function named
\texttt{gluLookAt()} allows to set the point of view, with its coordinates, 
the coordinates of the point to look at and its orientation.
\\
See \cite{opengl:gluLookAt} for further details.

\begin{lstlisting}[caption={OpenGL perspective example}, label={code:perspective}, frame=trBL]

  /* define the projection transformation */
  glMatrixMode(GL_PROJECTION);
  glLoadIdentity();
  gluPerspective(60, 624/442, 0.001, 100000);
  
  /* define the viewing transformation */
  glMatrixMode(GL_MODELVIEW);
  glLoadIdentity();
  gluLookAt(0.0, 0.0, 10.0,
            0.0, 0.0, 0.0,
            0.0, 1.0, 0.0);

\end{lstlisting}
