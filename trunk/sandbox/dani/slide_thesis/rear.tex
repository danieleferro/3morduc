\subsection{Intro}
\frame
{
  \frametitle{Intro}
  
  \emph{\textit{R.E.A.R.} is a framework for the
    development of \textit{virtual exocentric vision systems},
    regardless of some specif implementation detail.}
  \pause
  
  \vskip15pt


  \begin{block} {\alert{\texttt{How does it work}}}
    \begin{enumerate}

      \footnotesize

      \pause
      \item \texttt{take command input from user}
      \pause
      \item \texttt{send command to robot}
      \pause
      \item \texttt{retrieve robot position \& new egocentric image}
      \pause
      \item \texttt{add image to images collection}
      \pause
      \item \texttt{choose the proper image to use as background}
      \pause
      \item \texttt{move the camera accordingly to the image chosen}
      \pause
      \item \texttt{move robot in its current position}
      
    \end{enumerate}
      
  \end{block}
      
}

\subsection{Indipendence from lower details}
\frame
{
  \frametitle{Indipendence from lower details}
  
  \emph{\textit{R.E.A.R.}, written in C++,  exploits \alert{polymorphism} and \alert{inheritance} 
    object-oriented features to work regardless of spefic details.}
  
  \vskip10pt
  \pause

  \begin{block} {\alert{\texttt{Lower details}}}
    \begin{itemize}

    \item \alert{\textit{how to draw the robot 3D model}} \\
      \pause

      \vskip5pt
      \only<3>{\footnotesize{user specifies a subclass of \alert{\texttt{Robot}} class: \\

          \begin{description} [\texttt{DrawRobot()}]

            \item [\texttt{DrawRobot()}]
              encapsulates how draw the robot, through \textit{OpenGL} calls.

          \end{description}
        }}


      \pause

    \item \alert{\textit{how to choose the proper egocentric image}} \\
      \pause

      \vskip5pt
      \only<5>{\footnotesize{user specifies an implementation of \alert{\texttt{IImageSelector}} interface: \\

          \begin{description} [\texttt{ChoseImage()}]

            \item [\texttt{ChoseImage()}]
              encapsulates which image will be set as texture, given the current
              robot position and previous shot images set.
          \end{description}
        }}
          
      \pause

    \item \alert{\textit{how to send and retrieve data to/from robot}} \\

      \pause
      \vskip5pt
      \only<7>{\footnotesize{user specifies an implementation of \alert{\texttt{IDataLogic}} interface: \\

          \begin{description} [\texttt{RetrieveData()}]
            \item [\texttt{SelectImage()}]
              delegates exocentric image selection to proper object; 
            \item [\texttt{RetrieveData()}]
              encapsulates how to read new status (i.e. position and egocentric
              image) from robot;
            \item [\texttt{Command()}]
              encapsulates how to send a movement command (forward, backward,
              left, right) to robot.
          \end{description}
        }}

      \pause

    \end{itemize}
    
  \end{block}
}
