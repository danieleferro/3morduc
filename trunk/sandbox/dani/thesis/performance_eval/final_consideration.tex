\section{Final considerations}
\label{performance_evaluation:finalconsiderations}

According to us, two chief conclusion can be drawn.
\\
The former is that when the robot moves along straight lines
the optimal distance should be set to higher values, in order
to show the entire robot model and, hence, to give a \textit{fully}
exocentric vision.
\\
The latter one regards turnings. When robot performs hard turnings
the \textit{sweep metric algorithm} often selects the egocentric
point of view: changing from an exocentric to an egocentric vision
causes disturbance to users.
\\
To overcome this deficiency an evolution of the \textit{sweep metric
algorithm} was afterwards developed. It is named \textit{another sweep
metric algorithm} (proving the authors' lack of imagination) and
has been exposed in sections 
\ref{concr:iimageselector:another_sweep_metric_algorithm} and
\ref{concr:iimageselector:another_sweep_metric_class}.
\\
Even though the new algorithm has never been tested with users, its
benefits are immediately evident, in particular when robot moves
along a `square' path, because long straight distances are interspersed
with strict turns (angles greater than 45$\textdegree$ degrees).
