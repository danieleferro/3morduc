\section{Installing Soya}
This section will describe how to install soya under various OS.

Soya uses Python distutils for installation.

\begin{seealso}
  \seelink{http://oomadness.tuxfamily.org/en/soya/index.html}{Soya Homepage}{For the latest releases of Soya}
  \seelink{https://gna.org/cvs/?group=soya}{Soya CVS}{For instructions on obtaining Soya CVS}
\end{seealso}

\subsection{Software Requirements}

You need to have the following software installed:

\begin{itemize}
  \item Python 2.2 ( \url{http://python.org} also tested with Python 2.3 and 2.4)
  \item OpenGL ( \url{http://www.opengl.org/} )
  \item SDL ( \url{http://libsdl.org} )
  \item Cal3D ( \url{http://cal3d.sf.net} Version 0.10.0 )
  \item libFreeType2 ( \url{http://freetype.sf.net} )
  \item PIL ( \url{http://www.pythonware.com/products/pil/} )
  \item Pyrex 0.9.3 ( \url{http://www.cosc.canterbury.ac.nz/~greg/python/Pyrex} only for compiling Soya's CVS )
  \item Glew ( \url{http://glew.sf.net} only currently required for Soya CVS )
\end{itemize}

\begin{notice}
If you are using Linux and debian or ubuntu the package called libcal3d10 is not 
Cal3D version 0.10. You need to get the source and compile it. 
\end{notice}

\subsection{General installation}
Download the latest Soya release from \url{http://download.gna.org/soya/} or 
get the latest code from CVS ( instructions for CVS can be found at 
\url{https://gna.org/cvs/?group=soya} ). These instructions will presume 
you are using a downloaded release. Installation from CVS is covered later in this 
document. 

Once you have downloaded the release you will need to extract it:
\begin{verbatim}
\$ tar jxvf Soya-X.XX.tar.bz2
\end{verbatim}

Then enter the newly created directory:
\begin{verbatim}
$ cd Soya-X.XX
\end{verbatim}

The as the \emph{root} user execute:
\begin{verbatim}
$ python setup.py install
\end{verbatim}

If you encounter compilation problems you may need to edit \file{config.py}
to specify the locations of your libraries.

\subsection{Windows Installation}
Precompiled Windows binaries can be found at 
\url{http://thomas.paviot.free.fr/soya/}.
CVS builds and prebuilt dependacies can be found at 
\url{http://soya.literati.org/WindowsInstallers}.

\subsection{OSX Installation}
XXX TODO: who has a mac?

\subsection{Installation from CVS}
To compile from CVS you will need 
\ulink{Pyrex}{http://www.cosc.canterbury.ac.nz/~greg/python/Pyrex}.

To download CVS follow the instructions at 
\url{https://gna.org/cvs/?group=soya}.

To compile the source you need to execute:
\begin{verbatim}
$ python setup.py build
\end{verbatim}

Because of the source layout in Soya and limitiations of Pyrex, if 
you make any changes to the source you will often need to update the 
modification time of \file{_soya.pyx}. On Linux this can be done as 
follows:
\begin{verbatim}
$ touch _soya.pyx
\end{verbatim}

Another option is use the \file{--force} option ( this will rebuild
\emph{all the modules} so will take significantly longer ):
\begin{verbatim}
$ python setup.py build --force
\end{verbatim}

